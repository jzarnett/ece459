\input{configuration-plam}
\usepackage{soul}

\title{Lecture 27 --- Memory Profiling }

\author{Jeff Zarnett \\ \small \texttt{jzarnett@uwaterloo.ca}}
\institute{Department of Electrical and Computer Engineering \\
  University of Waterloo}
\date{\today}


\begin{document}

\begin{frame}
  \titlepage

 \end{frame}



\begin{frame}
\frametitle{\st{Memory Profiling} Return to Asgard}

\large
\begin{changemargin}{2cm}
So far: CPU profiling. 

Memory profiling is also a thing; \\
\qquad specifically heap profiling.

``Still Reachable'': not freed \& still have pointers, \\
\qquad but should have been freed?
\end{changemargin}

\end{frame}



\begin{frame}
\frametitle{\st{Memory Profiling} Return to Asgard}

\large
\begin{changemargin}{2cm}
As with queueing theory:\\
\qquad allocs $>$ frees $\Longrightarrow$ usage $\rightarrow \infty$

At least more paging, maybe total out-of-memory.

But! Memory isn't really lost: we could free it.

Our tool for this comes from the Valgrind tool suite.
\end{changemargin}

\end{frame}


\begin{frame}
\frametitle{Shieldmaiden to Thor}

\begin{center}
	\includegraphics[width=\textwidth]{images/Sif.jpg}
\end{center}

\end{frame}



\begin{frame}
\frametitle{Using Massif}

\Large
\begin{changemargin}{2cm}
What does Massif do? 

\begin{itemize}
\item How much heap memory is your program using?
\item How did this happen?
\end{itemize}

Next up: example from Massif docs.

\end{changemargin}

\end{frame}

\begin{frame}[fragile]
\frametitle{Example Allocation Program}

{\scriptsize
\begin{verbatim}
#include <stdlib.h>

void g ( void ) {
    malloc( 4000 );
}

void f ( void ) {
    malloc( 2000 );
    g();
}

int main ( void ) {
    int i;
    int* a[10];

    for ( i = 0; i < 10; i++ ) {
        a[i] = malloc( 1000 );
    }
    f();
    g();

    for ( i = 0; i < 10; i++ ) {
        free( a[i] );
    }
    return 0;
}
\end{verbatim}
}


\end{frame}

\begin{frame}[fragile]
\frametitle{Send in Sif}

\Large
\begin{changemargin}{2cm}
After we compile (remember \texttt{-g} for debug symbols), run the command:
\end{changemargin}
\vspace*{-4em}
{\scriptsize
\begin{verbatim}
jz@Loki:~/ece459$ valgrind --tool=massif ./massif
==25187== Massif, a heap profiler
==25187== Copyright (C) 2003-2013, and GNU GPL'd, by Nicholas Nethercote
==25187== Using Valgrind-3.10.1 and LibVEX; rerun with -h for copyright info
==25187== Command: ./massif
==25187== 
==25187== 
\end{verbatim}
}

\end{frame}


\begin{frame}
\frametitle{That Was Useful!!!}

\large
\begin{changemargin}{2cm}
What happened? 

\begin{enumerate}
\item The program ran slowly (because Valgrind!)

\item No summary data on the console \\
\hspace*{2em} (like memcheck or helgrind or cachegrind.)
\end{enumerate}

Weird. What we got instead was the file \texttt{massif.out.[PID]}.
\end{changemargin}

\end{frame}


\begin{frame}
\frametitle{Post-Processing}

\Large
\begin{changemargin}{2cm}
\texttt{massif.out.[PID]}:\\
\hspace*{2cm} plain text, sort of readable.

Better: \texttt{ms\_print}.

Which has nothing whatsoever to do with Microsoft. Promise.
\end{changemargin}

\end{frame}


\begin{frame}[fragile]
\frametitle{Post-Processed Output}
{\scriptsize
\begin{verbatim}

    KB
19.71^                                                                       #
     |                                                                       #
     |                                                                       #
     |                                                                       #
     |                                                                       #
     |                                                                       #
     |                                                                       #
     |                                                                       #
     |                                                                       #
     |                                                                       #
     |                                                                       #
     |                                                                       #
     |                                                                       #
     |                                                                       #
     |                                                                       #
     |                                                                       #
     |                                                                      :#
     |                                                                      :#
     |                                                                      :#
     |                                                                      :#
   0 +----------------------------------------------------------------------->ki
     0                                                                   111.9
\end{verbatim}
}
\end{frame}


\begin{frame}[fragile]
\frametitle{User Friendly, But Not Useful}

\Large
\begin{changemargin}{2cm}
For a long time, nothing happens, then\ldots kaboom! 

Why? We gave it a trivial program.

We should tell Massif to care more \\
about bytes than CPU cycles,\\
with \verb+--time-unit=B+.

Let's try that.
\end{changemargin}


\end{frame}

\begin{frame}[fragile]
\frametitle{ASCII Art ( \texttt{telnet towel.blinkenlights.nl} )}

{\scriptsize
\begin{verbatim}

    KB
19.71^                                               ### <- peak                
     |                                               #                        
     |                                               #  ::                    
     |                                               #  : ::: <- normal         
     |                                      :::::::::#  : :  ::               
     |                                      :        #  : :  : ::             
     |                                      :        #  : :  : : :::          
     |                                      :        #  : :  : : :  ::        
     |               detailed     :::::::::::        #  : :  : : :  : :::     
     |                     |      :         :        #  : :  : : :  : :  ::   
     |                     v  :::::         :        #  : :  : : :  : :  : :: 
     |                     @@@:   :         :        #  : :  : : :  : :  : : @
     |                   ::@  :   :         :        #  : :  : : :  : :  : : @
     |                :::: @  :   :         :        #  : :  : : :  : :  : : @
     |              :::  : @  :   :         :        #  : :  : : :  : :  : : @
     |            ::: :  : @  :   :         :        #  : :  : : :  : :  : : @
     |         :::: : :  : @  :   :         :        #  : :  : : :  : :  : : @
     |       :::  : : :  : @  :   :         :        #  : :  : : :  : :  : : @
     |    :::: :  : : :  : @  :   :         :        #  : :  : : :  : :  : : @
     |  :::  : :  : : :  : @  :   :         :        #  : :  : : :  : :  : : @
   0 +----------------------------------------------------------------------->KB
     0                                                                   29.63

\end{verbatim}
}

\end{frame}



\begin{frame}
\frametitle{Analyze the Art}

\Large
\begin{changemargin}{2cm}
OK! Massif took 25 snapshots.

\begin{itemize}
\item whenever there are appropriate allocation and deallocation statements, up to a configurable maximum. 
\end{itemize}

Long running program:\\ will toss some old data if necessary. 
\end{changemargin}
\end{frame}



%% \begin{frame}
%% \frametitle{Decode the Symbols}

%% \begin{itemize}
%% \item Normal: :

%% \item Detailed: @

%% \item Peak: \#
%% \end{itemize}

%% Peaks can be slightly inaccurate as they are recorded only at deallocation (and to speed up operations in general).

%% \end{frame}



\begin{frame}[fragile]
\frametitle{Normal Snapshots}

{\scriptsize
\begin{verbatim}
--------------------------------------------------------------------------------
  n        time(B)         total(B)   useful-heap(B) extra-heap(B)    stacks(B)
--------------------------------------------------------------------------------
  0              0                0                0             0            0
  1          1,016            1,016            1,000            16            0
  2          2,032            2,032            2,000            32            0
  3          3,048            3,048            3,000            48            0
  4          4,064            4,064            4,000            64            0
  5          5,080            5,080            5,000            80            0
  6          6,096            6,096            6,000            96            0
  7          7,112            7,112            7,000           112            0
  8          8,128            8,128            8,000           128            0
\end{verbatim}
}

\large
\begin{changemargin}{2cm}
time(B) column = time measured in allocations\\
(our choice of time unit on cmdline).

extra-heap(B) = internal fragmentation.

(Why are stacks all shown as 0?)
\end{changemargin}

\end{frame}


\begin{frame}[fragile]
\frametitle{Detailed Snapshots}

{\scriptsize
\begin{verbatim}
--------------------------------------------------------------------------------
  n        time(B)         total(B)   useful-heap(B) extra-heap(B)    stacks(B)
--------------------------------------------------------------------------------
  9          9,144            9,144            9,000           144            0
98.43% (9,000B) (heap allocation functions) malloc/new/new[], --alloc-fns, etc.
->98.43% (9,000B) 0x4005BB: main (massif.c:17)
\end{verbatim}
}

\large
\begin{changemargin}{2cm}
Now: where did heap allocations take place?

So far, all the allocations took place on line 17,\\
 which was \texttt{  a[i] = malloc( 1000 ); } \\
inside that for loop.
\end{changemargin}

\end{frame}


\begin{frame}[fragile]
\frametitle{Peak Snapshot (Trimmed)}


{\scriptsize
\begin{verbatim}
--------------------------------------------------------------------------------
  n        time(B)         total(B)   useful-heap(B) extra-heap(B)    stacks(B)
--------------------------------------------------------------------------------
 14         20,184           20,184           20,000           184            0
99.09% (20,000B) (heap allocation functions) malloc/new/new[], --alloc-fns, etc.
->49.54% (10,000B) 0x4005BB: main (massif.c:17)
| 
->39.64% (8,000B) 0x400589: g (massif.c:4)
| ->19.82% (4,000B) 0x40059E: f (massif.c:9)
| | ->19.82% (4,000B) 0x4005D7: main (massif.c:20)
| |   
| ->19.82% (4,000B) 0x4005DC: main (massif.c:22)
|   
->09.91% (2,000B) 0x400599: f (massif.c:8)
  ->09.91% (2,000B) 0x4005D7: main (massif.c:20)

\end{verbatim}
}

\large
\begin{changemargin}{2cm}
Massif found  all  allocations  and \\ distilled them to a tree structure.

We see not just where the \texttt{malloc} call happened, but also how we got there.
\end{changemargin}

\end{frame}


\begin{frame}[fragile]
\frametitle{``Is he dead?'' ``Terminated.''}

\large
\begin{changemargin}{2cm}
Termination gives a final output of what blocks remains allocated and where they come from. 

These point to memory leaks, incidentally, and Memcheck would not be amused.
\end{changemargin}
{\scriptsize
\begin{verbatim}
 24         30,344           10,024           10,000            24            0
99.76% (10,000B) (heap allocation functions) malloc/new/new[], --alloc-fns, etc.
->79.81% (8,000B) 0x400589: g (massif.c:4)
| ->39.90% (4,000B) 0x40059E: f (massif.c:9)
| | ->39.90% (4,000B) 0x4005D7: main (massif.c:20)
| |   
| ->39.90% (4,000B) 0x4005DC: main (massif.c:22)
|   
->19.95% (2,000B) 0x400599: f (massif.c:8)
| ->19.95% (2,000B) 0x4005D7: main (massif.c:20)
|   
->00.00% (0B) in 1+ places, all below ms_print's threshold (01.00%)
\end{verbatim}
}


\end{frame}


\begin{frame}[fragile]
\frametitle{Trust, but Verify}

\large
\begin{changemargin}{2cm}
Here's what Memcheck thinks:
\end{changemargin}

{\scriptsize
\begin{verbatim}
jz@Loki:~/ece459$ valgrind ./massif
==25775== Memcheck, a memory error detector
==25775== Copyright (C) 2002-2013, and GNU GPL'd, by Julian Seward et al.
==25775== Using Valgrind-3.10.1 and LibVEX; rerun with -h for copyright info
==25775== Command: ./massif
==25775== 
==25775== 
==25775== HEAP SUMMARY:
==25775==     in use at exit: 10,000 bytes in 3 blocks
==25775==   total heap usage: 13 allocs, 10 frees, 20,000 bytes allocated
==25775== 
==25775== LEAK SUMMARY:
==25775==    definitely lost: 10,000 bytes in 3 blocks
==25775==    indirectly lost: 0 bytes in 0 blocks
==25775==      possibly lost: 0 bytes in 0 blocks
==25775==    still reachable: 0 bytes in 0 blocks
==25775==         suppressed: 0 bytes in 0 blocks
==25775== Rerun with --leak-check=full to see details of leaked memory
==25775== 
==25775== For counts of detected and suppressed errors, rerun with: -v
==25775== ERROR SUMMARY: 0 errors from 0 contexts (suppressed: 0 from 0)
\end{verbatim}
}


\end{frame}


\begin{frame}
\frametitle{Valgrind (Memcheck) First}

\large
\begin{changemargin}{2cm}
Run valgrind (Memcheck) first and make it happy \\
before we go into figuring out where heap blocks are going with Massif. 

Okay, what to do with the information from Massif, anyway? 

Easy!
\begin{itemize}
\item Start with peak (worst case scenario) \\ and see where that takes you (if anywhere). 

\item You can probably identify some cases where memory is hanging around unnecessarily. 
\end{itemize}
\end{changemargin}


\end{frame}


\begin{frame}
\frametitle{Places to Look with Massif}

\large
\begin{changemargin}{2cm}
Memory usage climbing over a long period of time, perhaps slowly, but never decreasing---memory filling with junk? 

Large spikes in the graph---why so much allocation and deallocation in a short period?
\end{changemargin}
\end{frame}



\begin{frame}[fragile]
\frametitle{Other Massif-ly Useful Things}

\large
\begin{changemargin}{2cm}
\begin{itemize}
	\item stack allocation (\verb+--stacks=yes+).
	\item children of a process \\ (anything split off with \texttt{fork}) if desired.
	\item low level stuff: if going beyond \texttt{malloc}, \texttt{calloc}, \texttt{new}, etc. and using \texttt{mmap} or \texttt{brk} that is usually missed, can do profiling at page level (\verb+--pages-as-heap=yes+).
\end{itemize}
\end{changemargin}

\end{frame}




\begin{frame}
\frametitle{Live Demos}

\large
\begin{changemargin}{2cm}
As is often the case, \\ we have examined the tool on a trivial program. 

Let's see if we can do some\\
 live demos of Massif at work.
\end{changemargin}
\end{frame}


\end{document}

