\documentclass[letterpaper,10pt]{article}

\usepackage{titling}
\usepackage{listings}
\usepackage{url}
\usepackage{setspace}
\usepackage{subfig}
\usepackage{sectsty}
\usepackage{pdfpages}
\usepackage{colortbl}
\usepackage{multirow}
\usepackage{multicol}
\usepackage{relsize}
\usepackage{amsmath}
\usepackage{fancyvrb}
\usepackage{amsmath,amssymb,amsthm,graphicx,xspace}
\usepackage[titlenotnumbered,noend,noline]{algorithm2e}
\usepackage[compact]{titlesec}
\usepackage{XCharter}
\usepackage[T1]{fontenc}
\usepackage{enumitem}
\usepackage{tikz}
\usetikzlibrary{arrows,automata,shapes,trees,matrix,chains,scopes,positioning,calc}
\tikzstyle{block} = [rectangle, draw, fill=blue!20, 
    text width=2.5em, text centered, rounded corners, minimum height=2em]
\tikzstyle{bw} = [rectangle, draw, fill=blue!20, 
    text width=4em, text centered, rounded corners, minimum height=2em]

\newcommand{\CPP}{C\nolinebreak\hspace{-.05em}\raisebox{.4ex}{\tiny\bf +}\nolinebreak\hspace{-.10em}\raisebox{.4ex}{\tiny\bf +}}
\def\CPP{{C\nolinebreak[4]\hspace{-.05em}\raisebox{.4ex}{\tiny\bf ++}}}

\let\LaTeXtitle\title
\renewcommand{\title}[1]{\LaTeXtitle{\textsf{#1}}}


\addtolength{\oddsidemargin}{-1.000in}
\addtolength{\evensidemargin}{-0.500in}
\addtolength{\textwidth}{2.0in}
\addtolength{\topmargin}{-1.000in}
\addtolength{\textheight}{1.75in}
\addtolength{\parskip}{\baselineskip}
\setlength{\parindent}{0in}
\renewcommand{\baselinestretch}{1.5}

\singlespace


\begin{document}

\lecture{9 --- C++ Atomics, Compiler Hints, Restrict}{\term}{Patrick Lam and Jeff Zarnett}



\section*{The Compiler and You}
Making the compiler work for you is critical to programming for
performance. We'll therefore see some compiler implementation details
in this class. Understanding these details will help you reason about
how your code gets translated into machine code and thus executed.

\paragraph{Three Address Code.} Compiler analyses are much easier to
perform on simple expressions which have two operands and a
result---hence three addresses---rather than full expression trees.
Any good compiler will therefore convert a program's abstract syntax
tree into an intermediate, portable, three-address code before going
to a machine-specific backend.

Each statement represents one fundamental operation; we'll consider
these operations to be atomic. A typical statement looks like this:

\[ \qquad \mbox{result} := \mbox{operand$_1$}\:\mbox{operator}\:\mbox{operand$_2$} \]

Three-address code is useful for reasoning about data races. It is
also easier to read than assembly, as it separates out memory reads
and writes.

\paragraph{GIMPLE: \texttt{gcc}'s three-address code.} To see the GIMPLE representation 
of your code, pass {\tt gcc} the {\tt -fdump-tree-gimple} flag. You
can also see all of the three address code generated by the compiler;
use {\tt -fdump-tree-all}. You'll probably just be interested in the
optimized version.  

I suggest using GIMPLE to reason about your code at a low level
without having to read assembly. Let's take a few minutes to look at an example.


\subsection*{The {\tt restrict} qualifier} 
The {\tt restrict} qualifier on pointer {\tt p} tells
the compiler~\cite{cellperf} that it may assume that, in the scope of {\tt p},
the program will not use any other pointer {\tt q} to access the
data at {\tt *p}.

The {\tt restrict} qualifier is a feature introduced in C99: ``The
restrict type qualifier allows programs to be written so that
translators can produce significantly faster executables.''
  \begin{itemize}
    \item To request C99 in {\tt gcc}, use the {\tt -std=c99} flag.
  \end{itemize}

{\tt restrict} means: you are promising the
compiler that the pointer will never alias (another pointer will not
point to the same data) for the lifetime of the pointer.  Hence, two
pointers declared {\tt restrict} must never point to the same data.

In fact~\cite{cellperf} includes a contract that goes with the use of restrict:

\begin{quote}
I, [insert your name], a PROFESSIONAL or AMATEUR [circle one] programmer recognize that there are limits to what a compiler can do. I certify that, to the best of my knowledge, there are no magic elves or monkeys in the compiler which through the forces of fairy dust can always make code faster. I understand that there are some problems for which there is not enough information to solve. I hereby declare that given the opportunity to provide the compiler with sufficient information, perhaps through some key word, I will gladly use said keyword and not bitch and moan about how "the compiler should be doing this for me."

In this case, I promise that the pointer declared along with the restrict qualifier is not aliased. I certify that writes through this pointer will not effect the values read through any other pointer available in the same context which is also declared as restricted.

* Your agreement to this contract is implied by use of the restrict keyword ;)
\end{quote}

Of course, I highly recommend that you have your personal legal expert review this contract before you sign it. As I would for any contract. Contracts are serious business.

An example from Wikipedia:
\begin{verbatim}
  void updatePtrs(int* ptrA, int* ptrB, int* val) {
    *ptrA += *val;
    *ptrB += *val;
  }
\end{verbatim}
Would declaring all these pointers as {\tt restrict} generate better code?

Well, let's look at the GIMPLE.

\begin{verbatim}
void updatePtrs(int* ptrA, int* ptrB, int* val) {
 D.1609 = *ptrA;
 D.1610 = *val;
 D.1611 = D.1609 + D.1610;
 *ptrA = D.1611;
 D.1612 = *ptrB;
 D.1610 = *val;
 D.1613 = D.1612 + D.1610;
 *ptrB = D.1613;
}
\end{verbatim}

Now we can answer the question: ``Could any operation be left out if
all the pointers didn't overlap?''

\begin{itemize}
\item If {\tt ptrA} and {\tt val} are not equal, you don't have to
      reload the data on {\bf line 7}.
\item Otherwise, you would: there might be a call, somewhere:\\\verb+    updatePtrs(&x, &y, &x);+
\end{itemize}

Hence, this set of annotations allows optimization:
\begin{verbatim}
    void updatePtrs(int* restrict ptrA, 
                    int* restrict ptrB,
                    int* restrict val)
\end{verbatim}
Note: you can get the optimization by just declaring {\tt ptrA} and
      {\tt val} as {\tt restrict}; {\tt ptrB} isn't needed for this optimization

\paragraph{Summary of {\tt restrict}.}
Use {\tt restrict} whenever you know the pointer will not alias
another pointer (also declared {\tt restrict}).

It's hard for the compiler to infer pointer aliasing information;
it's easier for you to specify it. If the compiler has this information,
it can better optimize your code; in the body of a critical loop, that
can result in better performance.

A caveat: don't lie to the compiler, or you will get undefined behaviour.

Aside: {\tt restrict} is not the same as {\tt const}. {\tt const} data can still be
changed through an alias.



\bibliographystyle{alpha}
\bibliography{459}


\end{document}
