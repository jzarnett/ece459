
\documentclass[letterpaper,hide notes,xcolor={table,svgnames},pdftex,10pt]{beamer}
\def\showexamples{t}


%\usepackage[svgnames]{xcolor}

%% Demo talk
%\documentclass[letterpaper,notes=show]{beamer}

\usecolortheme{crane}
\setbeamertemplate{navigation symbols}{}

\usetheme{MyPittsburgh}
%\usetheme{Frankfurt}

%\usepackage{tipa}

\usepackage{hyperref}
\usepackage{graphicx,xspace}
\usepackage[normalem]{ulem}
\usepackage{multicol}

\newcommand\SF[1]{$\bigstar$\footnote{SF: #1}}

\usepackage[default]{sourcesanspro}
\usepackage[T1]{fontenc}

\newcounter{tmpnumSlide}
\newcounter{tmpnumNote}

% old question code
%\newcommand\question[1]{{$\bigstar$ \small \onlySlide{2}{#1}}}
% \newcommand\nquestion[1]{\ifdefined \presentationonly \textcircled{?} \fi \note{\par{\Large \textbf{?}} #1}}
% \newcommand\nanswer[1]{\note{\par{\Large \textbf{A}} #1}}


 \newcommand\mnote[1]{%
   \addtocounter{tmpnumSlide}{1}
   \ifdefined\showcues {~\tiny\fbox{\arabic{tmpnumSlide}}}\fi
   \note{\setlength{\parskip}{1ex}\addtocounter{tmpnumNote}{1}\textbf{\Large \arabic{tmpnumNote}:} {#1\par}}}

\newcommand\mmnote[1]{\note{\setlength{\parskip}{1ex}#1\par}}

%\newcommand\mnote[2][]{\ifdefined\handoutwithnotes {~\tiny\fbox{#1}}\fi
% \note{\setlength{\parskip}{1ex}\textbf{\Large #1:} #2\par}}

%\newcommand\mnote[2][]{{\tiny\fbox{#1}} \note{\setlength{\parskip}{1ex}\textbf{\Large #1:} #2\par}}

\newcommand\mquestion[2]{{~\color{red}\fbox{?}}\note{\setlength{\parskip}{1ex}\par{\Large \textbf{?}} #1} \note{\setlength{\parskip}{1ex}\par{\Large \textbf{A}} #2\par}\ifdefined \presentationonly \pause \fi}

\newcommand\blackboard[1]{%
\ifdefined   \showblackboard
  {#1}
  \else {\begin{center} \fbox{\colorbox{blue!30}{%
         \begin{minipage}{.95\linewidth}%
           \hspace{\stretch{1}} Some space intentionally left blank; done at the blackboard.%
         \end{minipage}}}\end{center}}%
         \fi%
}



%\newcommand\q{\tikz \node[thick,color=black,shape=circle]{?};}
%\newcommand\q{\ifdefined \presentationonly \textcircled{?} \fi}

\usepackage{listings}
\lstset{%
  keywordstyle=\bfseries,
  aboveskip=15pt,
  belowskip=15pt,
  captionpos=b,
  identifierstyle=\ttfamily,
  escapeinside={(*@}{@*)},
  stringstyle=\ttfamiliy,
  frame=lines,
  numbers=left, basicstyle=\scriptsize, numberstyle=\tiny, stepnumber=0, numbersep=2pt}

\usepackage{siunitx}
\newcommand\sius[1]{\num[group-separator = {,}]{#1}\si{\micro\second}}
\newcommand\sims[1]{\num[group-separator = {,}]{#1}\si{\milli\second}}
\newcommand\sins[1]{\num[group-separator = {,}]{#1}\si{\nano\second}}
\sisetup{group-separator = {,}, group-digits = true}

%% -------------------- tikz --------------------
\usepackage{tikz}
\usetikzlibrary{positioning}
\usetikzlibrary{arrows,backgrounds,automata,decorations.shapes,decorations.pathmorphing,decorations.markings,decorations.text}

\tikzstyle{place}=[circle,draw=blue!50,fill=blue!20,thick, inner sep=0pt,minimum size=6mm]
\tikzstyle{transition}=[rectangle,draw=black!50,fill=black!20,thick, inner sep=0pt,minimum size=4mm]

\tikzstyle{block}=[rectangle,draw=black, thick, inner sep=5pt]
\tikzstyle{bullet}=[circle,draw=black, fill=black, thin, inner sep=2pt]

\tikzstyle{pre}=[<-,shorten <=1pt,>=stealth',semithick]
\tikzstyle{post}=[->,shorten >=1pt,>=stealth',semithick]
\tikzstyle{bi}=[<->,shorten >=1pt,shorten <=1pt, >=stealth',semithick]

\tikzstyle{mut}=[-,>=stealth',semithick]

\tikzstyle{treereset}=[dashed,->, shorten >=1pt,>=stealth',thin]

\usepackage{ifmtarg}
\usepackage{xifthen}
\makeatletter
% new counter to now which frame it is within the sequence
\newcounter{multiframecounter}
% initialize buffer for previously used frame title
\gdef\lastframetitle{\textit{undefined}}
% new environment for a multi-frame
\newenvironment{multiframe}[1][]{%
\ifthenelse{\isempty{#1}}{%
% if no frame title was set via optional parameter,
% only increase sequence counter by 1
\addtocounter{multiframecounter}{1}%
}{%
% new frame title has been provided, thus
% reset sequence counter to 1 and buffer frame title for later use
\setcounter{multiframecounter}{1}%
\gdef\lastframetitle{#1}%
}%
% start conventional frame environment and
% automatically set frame title followed by sequence counter
\begin{frame}%
\frametitle{\lastframetitle~{\normalfont(\arabic{multiframecounter})}}%
}{%
\end{frame}%
}
\makeatother

\makeatletter
\newdimen\tu@tmpa%
\newdimen\ydiffl%
\newdimen\xdiffl%
\newcommand\ydiff[2]{%
    \coordinate (tmpnamea) at (#1);%
    \coordinate (tmpnameb) at (#2);%
    \pgfextracty{\tu@tmpa}{\pgfpointanchor{tmpnamea}{center}}%
    \pgfextracty{\ydiffl}{\pgfpointanchor{tmpnameb}{center}}%
    \advance\ydiffl by -\tu@tmpa%
}
\newcommand\xdiff[2]{%
    \coordinate (tmpnamea) at (#1);%
    \coordinate (tmpnameb) at (#2);%
    \pgfextractx{\tu@tmpa}{\pgfpointanchor{tmpnamea}{center}}%
    \pgfextractx{\xdiffl}{\pgfpointanchor{tmpnameb}{center}}%
    \advance\xdiffl by -\tu@tmpa%
}
\makeatother
\newcommand{\copyrightbox}[3][r]{%
\begin{tikzpicture}%
\node[inner sep=0pt,minimum size=2em](ciimage){#2};
\usefont{OT1}{phv}{n}{n}\fontsize{4}{4}\selectfont
\ydiff{ciimage.south}{ciimage.north}
\xdiff{ciimage.west}{ciimage.east}
\ifthenelse{\equal{#1}{r}}{%
\node[inner sep=0pt,right=1ex of ciimage.south east,anchor=north west,rotate=90]%
{\raggedleft\color{black!50}\parbox{\the\ydiffl}{\raggedright{}#3}};%
}{%
\ifthenelse{\equal{#1}{l}}{%
\node[inner sep=0pt,right=1ex of ciimage.south west,anchor=south west,rotate=90]%
{\raggedleft\color{black!50}\parbox{\the\ydiffl}{\raggedright{}#3}};%
}{%
\node[inner sep=0pt,below=1ex of ciimage.south west,anchor=north west]%
{\raggedleft\color{black!50}\parbox{\the\xdiffl}{\raggedright{}#3}};%
}
}
\end{tikzpicture}
}


%% --------------------

%\usepackage[excludeor]{everyhook}
%\PushPreHook{par}{\setbox0=\lastbox\llap{MUH}}\box0}

%\vspace*{\stretch{1}

%\setbox0=\lastbox \llap{\textbullet\enskip}\box0}

\setlength{\parskip}{\fill}

\newcommand\noskips{\setlength{\parskip}{1ex}}
\newcommand\doskips{\setlength{\parskip}{\fill}}

\newcommand\xx{\par\vspace*{\stretch{1}}\par}
\newcommand\xxs{\par\vspace*{2ex}\par}
\newcommand\tuple[1]{\langle #1 \rangle}
\newcommand\code[1]{{\sf \footnotesize #1}}
\newcommand\ex[1]{\uline{Example:} \ifdefined \presentationonly \pause \fi
  \ifdefined\showexamples#1\xspace\else{\uline{\hspace*{2cm}}}\fi}

\newcommand\ceil[1]{\lceil #1 \rceil}


\AtBeginSection[]
{
   \begin{frame}
       \frametitle{Outline}
       \tableofcontents[currentsection]
   \end{frame}
}



\pgfdeclarelayer{edgelayer}
\pgfdeclarelayer{nodelayer}
\pgfsetlayers{edgelayer,nodelayer,main}

\tikzstyle{none}=[inner sep=0pt]
\tikzstyle{rn}=[circle,fill=Red,draw=Black,line width=0.8 pt]
\tikzstyle{gn}=[circle,fill=Lime,draw=Black,line width=0.8 pt]
\tikzstyle{yn}=[circle,fill=Yellow,draw=Black,line width=0.8 pt]
\tikzstyle{empty}=[circle,fill=White,draw=Black]
\tikzstyle{bw} = [rectangle, draw, fill=blue!20, 
    text width=4em, text centered, rounded corners, minimum height=2em]
    
    \newcommand{\CcNote}[1]{% longname
	This work is licensed under the \textit{Creative Commons #1 3.0 License}.%
}
\newcommand{\CcImageBy}[1]{%
	\includegraphics[scale=#1]{creative_commons/cc_by_30.pdf}%
}
\newcommand{\CcImageSa}[1]{%
	\includegraphics[scale=#1]{creative_commons/cc_sa_30.pdf}%
}
\newcommand{\CcImageNc}[1]{%
	\includegraphics[scale=#1]{creative_commons/cc_nc_30.pdf}%
}
\newcommand{\CcGroupBySa}[2]{% zoom, gap
	\CcImageBy{#1}\hspace*{#2}\CcImageNc{#1}\hspace*{#2}\CcImageSa{#1}%
}
\newcommand{\CcLongnameByNcSa}{Attribution-NonCommercial-ShareAlike}

\newenvironment{changemargin}[1]{% 
  \begin{list}{}{% 
    \setlength{\topsep}{0pt}% 
    \setlength{\leftmargin}{#1}% 
    \setlength{\rightmargin}{1em}
    \setlength{\listparindent}{\parindent}% 
    \setlength{\itemindent}{\parindent}% 
    \setlength{\parsep}{\parskip}% 
  }% 
  \item[]}{\end{list}} 




\title{Lecture 25 --- Using Google Performance Tools }

\author{Patrick Lam \\ \small \texttt{p.lam@ece.uwaterloo.ca}}
\institute{Department of Electrical and Computer Engineering \\
  University of Waterloo}
\date{\today}


\begin{document}

\begin{frame}
  \titlepage

 \end{frame}

%%%%%%%%%%%%%%%%%%%%%%%%%%%%%%%%%%%%%%%%%%%%%%%%%%%%%%%%%%%%%%%%%%%%%%%%%%%%%%%%

\part{gperftools}
\frame{\partpage}

%%%%%%%%%%%%%%%%%%%%%%%%%%%%%%%%%%%%%%%%%%%%%%%%%%%%%%%%%%%%%%%%%%%%%%%%%%%%%%%%
\begin{frame}
  \frametitle{Introduction to gperftools}

  
    Google Performance Tools include:
      
      \begin{itemize}
        \item CPU profiler.
        \item Heap profiler.
        \item Heap checker.
        \item Faster (multithreaded) {\tt malloc}.
      \end{itemize}
~\\[1em]
     We'll mostly use the CPU profiler:
      \begin{itemize}
        \item purely statistical sampling;
        \item no recompilation; at most linking; and
        \item built-in visual output.
      \end{itemize}
  
\end{frame}
%%%%%%%%%%%%%%%%%%%%%%%%%%%%%%%%%%%%%%%%%%%%%%%%%%%%%%%%%%%%%%%%%%%%%%%%%%%%%%%%

%%%%%%%%%%%%%%%%%%%%%%%%%%%%%%%%%%%%%%%%%%%%%%%%%%%%%%%%%%%%%%%%%%%%%%%%%%%%%%%%
\begin{frame}[fragile]
  \frametitle{Google Perf Tools profiler usage}

  
    You can use the profiler without any recompilation.
      \begin{itemize}
        \item Not recommended---worse data.
      \end{itemize}

  \begin{lstlisting}
LD_PRELOAD="/usr/lib/libprofiler.so" \
CPUPROFILE=test.prof ./test
  \end{lstlisting}

     The other option is to link to the profiler:
      \begin{itemize}
        \item {\tt -lprofiler}
      \end{itemize}
    Both options read the {\tt CPUPROFILE} environment variable:
      \begin{itemize}
        \item states the location to write the profile data.
      \end{itemize}
  
\end{frame}
%%%%%%%%%%%%%%%%%%%%%%%%%%%%%%%%%%%%%%%%%%%%%%%%%%%%%%%%%%%%%%%%%%%%%%%%%%%%%%%%

%%%%%%%%%%%%%%%%%%%%%%%%%%%%%%%%%%%%%%%%%%%%%%%%%%%%%%%%%%%%%%%%%%%%%%%%%%%%%%%%
\begin{frame}[fragile]
  \frametitle{Other Usage}

  
     You can use the profiling library directly as well:
      \begin{itemize}
        \item {\tt \#include <gperftools/profiler.h>}
      \end{itemize}
     Bracket code you want profiled with:
      \begin{itemize}
        \item {\tt ProfilerStart()}
        \item {\tt ProfilerStop()}
      \end{itemize}~\\
    
    You can change the sampling frequency with the
      {\tt CPUPROFILE\_FREQUENCY} environment variable.
      \begin{itemize}
        \item {\bf Default value:} 100
      \end{itemize}
  
\end{frame}
%%%%%%%%%%%%%%%%%%%%%%%%%%%%%%%%%%%%%%%%%%%%%%%%%%%%%%%%%%%%%%%%%%%%%%%%%%%%%%%%

%%%%%%%%%%%%%%%%%%%%%%%%%%%%%%%%%%%%%%%%%%%%%%%%%%%%%%%%%%%%%%%%%%%%%%%%%%%%%%%%
\begin{frame}[fragile]
  \frametitle{{\tt pprof} Usage}


    Like {\tt gprof}, it will analyze profiling results.

  \begin{lstlisting}
% pprof test test.prof
    Enters "interactive" mode
% pprof --text test test.prof
    Outputs one line per procedure
% pprof --gv test test.prof
     Displays annotated call-graph via 'gv'
% pprof --gv --focus=Mutex test test.prof
    Restricts to code paths including a .*Mutex.* entry
% pprof --gv --focus=Mutex --ignore=string test test.prof
    Code paths including Mutex but not string
% pprof --list=getdir test test.prof
    (Per-line) annotated source listing for getdir()
% pprof --disasm=getdir test test.prof
    (Per-PC) annotated disassembly for getdir()
  \end{lstlisting}

    Can also output {\tt dot}, {\tt ps}, {\tt pdf} or {\tt gif} instead of
      {\tt gv}.
  

\end{frame}
%%%%%%%%%%%%%%%%%%%%%%%%%%%%%%%%%%%%%%%%%%%%%%%%%%%%%%%%%%%%%%%%%%%%%%%%%%%%%%%%

%%%%%%%%%%%%%%%%%%%%%%%%%%%%%%%%%%%%%%%%%%%%%%%%%%%%%%%%%%%%%%%%%%%%%%%%%%%%%%%%
\begin{frame}[fragile]
  \frametitle{Text Output}


    Similar to the flat profile in {\tt gprof}

  \begin{lstlisting}
jon@riker examples master % pprof --text test test.prof 
Using local file test.
Using local file test.prof.
Removing killpg from all stack traces.
Total: 300 samples
      95  31.7%  31.7%      102  34.0% int_power
      58  19.3%  51.0%       58  19.3% float_power
      51  17.0%  68.0%       96  32.0% float_math_helper
      50  16.7%  84.7%      137  45.7% int_math_helper
      18   6.0%  90.7%      131  43.7% float_math
      14   4.7%  95.3%      159  53.0% int_math
      14   4.7% 100.0%      300 100.0% main
       0   0.0% 100.0%      300 100.0% __libc_start_main
       0   0.0% 100.0%      300 100.0% _start
  \end{lstlisting}
  

\end{frame}
%%%%%%%%%%%%%%%%%%%%%%%%%%%%%%%%%%%%%%%%%%%%%%%%%%%%%%%%%%%%%%%%%%%%%%%%%%%%%%%%

%%%%%%%%%%%%%%%%%%%%%%%%%%%%%%%%%%%%%%%%%%%%%%%%%%%%%%%%%%%%%%%%%%%%%%%%%%%%%%%%
\begin{frame}
  \frametitle{Text Output Explained}

  
    Columns, from left to right:\\[1em]

    Number of checks (samples) in this function.\\
    Percentage of checks in this function.
      \begin{itemize}
        \item Same as {\bf time} in {\tt gprof}.\\[1em]
      \end{itemize}
    Percentage of checks in the functions printed so far.
      \begin{itemize}
        \item Equivalent to {\bf cumulative} (but in \%).\\[1em]
      \end{itemize}
    Number of checks in this function and its callees.\\
    Percentage of checks in this function and its callees.\\
    Function name.
  
\end{frame}
%%%%%%%%%%%%%%%%%%%%%%%%%%%%%%%%%%%%%%%%%%%%%%%%%%%%%%%%%%%%%%%%%%%%%%%%%%%%%%%%

%%%%%%%%%%%%%%%%%%%%%%%%%%%%%%%%%%%%%%%%%%%%%%%%%%%%%%%%%%%%%%%%%%%%%%%%%%%%%%%%
\begin{frame}
  \frametitle{Graphical Output}

  \begin{center}
    \includegraphics[width=0.5\textwidth]{images/pprof-test-big.png}
  \end{center}
\end{frame}
%%%%%%%%%%%%%%%%%%%%%%%%%%%%%%%%%%%%%%%%%%%%%%%%%%%%%%%%%%%%%%%%%%%%%%%%%%%%%%%%

%%%%%%%%%%%%%%%%%%%%%%%%%%%%%%%%%%%%%%%%%%%%%%%%%%%%%%%%%%%%%%%%%%%%%%%%%%%%%%%%
\begin{frame}
  \frametitle{Graphical Output Explained}

  
  Output was too small to read on the slide.

  \begin{itemize}
    \item Shows the same numbers as the text output.
    \item Directed edges denote function calls.
    \item Note: number of samples in callees = \\
      \qquad number in ``this function + callees'' - \\
      \qquad number in ``this function''.\\
    \item {\bf Example:}\\
{{\tt float\_math\_helper}, 51 (local) of 96 (cumulative)} \\
      96 - 51 = 45 (callees)
      \begin{itemize}
        \item callee {\tt int\_power} = 7 (bogus)
        \item callee {\tt float\_power} = 38
        \item callees total = 45
      \end{itemize}

  \end{itemize}
  
\end{frame}
%%%%%%%%%%%%%%%%%%%%%%%%%%%%%%%%%%%%%%%%%%%%%%%%%%%%%%%%%%%%%%%%%%%%%%%%%%%%%%%%

%%%%%%%%%%%%%%%%%%%%%%%%%%%%%%%%%%%%%%%%%%%%%%%%%%%%%%%%%%%%%%%%%%%%%%%%%%%%%%%%
\begin{frame}
  \frametitle{Things You May Notice}

  
    Call graph is not exact.
    
      \begin{itemize}
        \item In fact, it shows many bogus relations which clearly don't exist.
        \item For instance, we know that there are no cross-calls between {\tt int} and {\tt float} functions.
      \end{itemize}~\\

    As with {\tt gprof}, optimizations will change the
      graph.\\[1em]

    You'll probably want to look at the text profile first, then use the
      {\tt --focus} flag to look at individual functions.
  
\end{frame}
%%%%%%%%%%%%%%%%%%%%%%%%%%%%%%%%%%%%%%%%%%%%%%%%%%%%%%%%%%%%%%%%%%%%%%%%%%%%%%%%

% How to Read
% http://www.cs.utah.edu/dept/old/texinfo/as/gprof.html
% http://www.civilnet.cn/book/kernel/GNU.Linux.Application.Programming/LiB0055.html
% http://ececmpsysweb.groups.et.byu.net/cmpsys.2004.winter/citizenship/Bryan_Wheeler/Profiling_Tutorial.html

% gprof
% OProfile
% Valgrind
% Intel VTune
% AMD CodeAnalyst

\part{System profiling: oprofile,~perf,~DTrace,~WAIT}
\frame{\partpage}

%%%%%%%%%%%%%%%%%%%%%%%%%%%%%%%%%%%%%%%%%%%%%%%%%%%%%%%%%%%%%%%%%%%%%%%%%%%%%%%%
\begin{frame}
  \frametitle{Introduction: oprofile}


    \url{http://oprofile.sourceforge.net}\\[1em]

    Sampling-based tool.\\[1em]
    
    Uses CPU performance counters.\\[1em]

    Tracks currently-running function;\\
    records profiling data for every application run.\\[1em]

    Can work system-wide (across processes).\\[1em]

    Technology: Linux Kernel Performance Events\\ (formerly a Linux kernel module).

\end{frame}
%%%%%%%%%%%%%%%%%%%%%%%%%%%%%%%%%%%%%%%%%%%%%%%%%%%%%%%%%%%%%%%%%%%%%%%%%%%%%%%%

%%%%%%%%%%%%%%%%%%%%%%%%%%%%%%%%%%%%%%%%%%%%%%%%%%%%%%%%%%%%%%%%%%%%%%%%%%%%%%%%
\begin{frame}[fragile]
  \frametitle{Setting up {\tt oprofile}}


  Must run as root to use system-wide, otherwise can use per-process.

  \begin{lstlisting}
% sudo opcontrol \
     --vmlinux=/usr/src/linux-3.2.7-1-ARCH/vmlinux
% echo 0 | sudo tee /proc/sys/kernel/nmi_watchdog
% sudo opcontrol --start
Using default event: CPU_CLK_UNHALTED:100000:0:1:1
Using 2.6+ OProfile kernel interface.
Reading module info.
Using log file /var/lib/oprofile/samples/oprofiled.log
Daemon started.
Profiler running.
  \end{lstlisting}

  Per-process:
\begin{lstlisting}
[plam@lynch nm-morph]$ operf ./test_harness
operf: Profiler started

Profiling done.
\end{lstlisting}%$


\end{frame}
%%%%%%%%%%%%%%%%%%%%%%%%%%%%%%%%%%%%%%%%%%%%%%%%%%%%%%%%%%%%%%%%%%%%%%%%%%%%%%%%

%%%%%%%%%%%%%%%%%%%%%%%%%%%%%%%%%%%%%%%%%%%%%%%%%%%%%%%%%%%%%%%%%%%%%%%%%%%%%%%%
\begin{frame}[fragile]
  \frametitle{{\tt oprofile} Usage (1)}
  

  Pass your executable to {\tt opreport}.

  \begin{lstlisting}
% sudo opreport -l ./test    
CPU: Intel Core/i7, speed 1595.78 MHz (estimated)
Counted CPU_CLK_UNHALTED events (Clock cycles when not
halted) with a unit mask of 0x00 (No unit mask) count 100000
samples  %        symbol name
7550     26.0749  int_math_helper
5982     20.6596  int_power
5859     20.2348  float_power
3605     12.4504  float_math
3198     11.0447  int_math
2601      8.9829  float_math_helper
160       0.5526  main
  \end{lstlisting}
  
    If you have debug symbols ({\tt -g}) you could use:

  \begin{lstlisting}
% sudo opannotate --source \
--output-dir=/path/to/annotated-source /path/to/mybinary
  \end{lstlisting}

\end{frame}
%%%%%%%%%%%%%%%%%%%%%%%%%%%%%%%%%%%%%%%%%%%%%%%%%%%%%%%%%%%%%%%%%%%%%%%%%%%%%%%%

%%%%%%%%%%%%%%%%%%%%%%%%%%%%%%%%%%%%%%%%%%%%%%%%%%%%%%%%%%%%%%%%%%%%%%%%%%%%%%%%
\begin{frame}[fragile]
  \frametitle{{\tt oprofile} Usage (2)}
  

    Use {\tt opreport} by itself for a whole-system view.\\
    You can also reset and stop the profiling.

  \begin{lstlisting}
% sudo opcontrol --reset 
Signalling daemon... done
% sudo opcontrol --stop
Stopping profiling.
  \end{lstlisting}

\end{frame}
%%%%%%%%%%%%%%%%%%%%%%%%%%%%%%%%%%%%%%%%%%%%%%%%%%%%%%%%%%%%%%%%%%%%%%%%%%%%%%%%

%%%%%%%%%%%%%%%%%%%%%%%%%%%%%%%%%%%%%%%%%%%%%%%%%%%%%%%%%%%%%%%%%%%%%%%%%%%%%%%%
\begin{frame}
  \frametitle{Perf: Introduction}


    \url{https://perf.wiki.kernel.org/index.php/Tutorial}\\[1em]

    Interface to Linux kernel built-in sampling-based profiling.\\
    Per-process, per-CPU, or system-wide.\\
    Can even report the cost of each line of code.

\end{frame}
%%%%%%%%%%%%%%%%%%%%%%%%%%%%%%%%%%%%%%%%%%%%%%%%%%%%%%%%%%%%%%%%%%%%%%%%%%%%%%%%

%%%%%%%%%%%%%%%%%%%%%%%%%%%%%%%%%%%%%%%%%%%%%%%%%%%%%%%%%%%%%%%%%%%%%%%%%%%%%%%%
\begin{frame}[fragile]
  \frametitle{Perf: Usage Example}

On last year's Assignment 3 code:
\begin{lstlisting}[basicstyle=\tiny]
[plam@lynch nm-morph]$ perf stat ./test_harness

 Performance counter stats for './test_harness':

       6562.501429 task-clock                #    0.997 CPUs utilized          
               666 context-switches          #    0.101 K/sec                  
                 0 cpu-migrations            #    0.000 K/sec                  
             3,791 page-faults               #    0.578 K/sec                  
    24,874,267,078 cycles                    #    3.790 GHz                     [83.32%]
    12,565,457,337 stalled-cycles-frontend   #   50.52% frontend cycles idle    [83.31%]
     5,874,853,028 stalled-cycles-backend    #   23.62% backend  cycles idle    [66.63%]
    33,787,408,650 instructions              #    1.36  insns per cycle        
                                             #    0.37  stalled cycles per insn [83.32%]
     5,271,501,213 branches                  #  803.276 M/sec                   [83.38%]
       155,568,356 branch-misses             #    2.95% of all branches         [83.36%]

       6.580225847 seconds time elapsed
\end{lstlisting} %$
\end{frame}
%%%%%%%%%%%%%%%%%%%%%%%%%%%%%%%%%%%%%%%%%%%%%%%%%%%%%%%%%%%%%%%%%%%%%%%%%%%%%%%%

%%%%%%%%%%%%%%%%%%%%%%%%%%%%%%%%%%%%%%%%%%%%%%%%%%%%%%%%%%%%%%%%%%%%%%%%%%%%%%%%
\begin{frame}[fragile]
  \frametitle{Perf: Source-level Analysis}


perf can tell you which instructions are taking time, or which lines of code.\\[1em]

Compile with {\tt -ggdb} to enable source code viewing.

\begin{lstlisting}
% perf record ./test_harness
% perf annotate
\end{lstlisting}

{\tt perf annotate} is interactive. Play around with it.


\end{frame}
%%%%%%%%%%%%%%%%%%%%%%%%%%%%%%%%%%%%%%%%%%%%%%%%%%%%%%%%%%%%%%%%%%%%%%%%%%%%%%%%

%%%%%%%%%%%%%%%%%%%%%%%%%%%%%%%%%%%%%%%%%%%%%%%%%%%%%%%%%%%%%%%%%%%%%%%%%%%%%%%%
\begin{frame}
  \frametitle{DTrace: Introduction}


    \url{http://queue.acm.org/detail.cfm?id=1117401}\\[1em]

    Intrumentation-based tool.\\
    System-wide.\\
    Meant to be used on production systems. (Eh?)\\[1em]
     \only<2>{
     {\small (Typical instrumentation can have a slowdown of 100x (Valgrind).)}\\
     Design goals:\\
\begin{enumerate} 
\item No overhead when not in use;
\item Guarantee safety---must not crash\\ \qquad (strict limits on expressiveness of probes).
\end{enumerate}
     }

\end{frame}
%%%%%%%%%%%%%%%%%%%%%%%%%%%%%%%%%%%%%%%%%%%%%%%%%%%%%%%%%%%%%%%%%%%%%%%%%%%%%%%%

%%%%%%%%%%%%%%%%%%%%%%%%%%%%%%%%%%%%%%%%%%%%%%%%%%%%%%%%%%%%%%%%%%%%%%%%%%%%%%%%
\begin{frame}
  \frametitle{DTrace: Operation}


    How does DTrace achieve 0 overhead?\\
\begin{itemize}
    \item only when activated, dynamically rewrites code by placing a branch to
      instrumentation code.
\end{itemize}

    Uninstrumented: runs as if nothing changed.\\[1em]

    Most instrumentation: at function entry or exit points.\\
    You can also instrument kernel functions, locking, instrument-based
      on other events.\\[1em]

    Can express sampling as instrumentation-based events also.

\end{frame}
%%%%%%%%%%%%%%%%%%%%%%%%%%%%%%%%%%%%%%%%%%%%%%%%%%%%%%%%%%%%%%%%%%%%%%%%%%%%%%%%

%%%%%%%%%%%%%%%%%%%%%%%%%%%%%%%%%%%%%%%%%%%%%%%%%%%%%%%%%%%%%%%%%%%%%%%%%%%%%%%%
\begin{frame}[fragile]
  \frametitle{DTrace Example}


  You write this:

  \begin{lstlisting}
syscall::read:entry {
    self->t = timestamp;
}

syscall::read:return
/self->t/ {
    printf("%d/%d spent %d nsecs in read\n"
           pid, tid, timestamp - self->t);
}
  \end{lstlisting}

    {\tt t} is a thread-local variable.\\
    This code prints how long each call to {\tt read} takes, along with
      context.\\[1em]
    To ensure safety, DTrace limits what you write; e.g. no loops.
      \begin{itemize}
        \item (Hence, no infinite loops!)
      \end{itemize}


\end{frame}
%%%%%%%%%%%%%%%%%%%%%%%%%%%%%%%%%%%%%%%%%%%%%%%%%%%%%%%%%%%%%%%%%%%%%%%%%%%%%%%%

%%%%%%%%%%%%%%%%%%%%%%%%%%%%%%%%%%%%%%%%%%%%%%%%%%%%%%%%%%%%%%%%%%%%%%%%%%%%%%%%
\begin{frame}[fragile]
  \frametitle{Other Tools}


    AMD CodeAnalyst---based on oprofile; leverages AMD processor features.\\[1em]

    WAIT
      \begin{itemize}
        \item IBM's tool tells you what operations your JVM is waiting on while
          idle.
        \item Non-free and not available.
      \end{itemize}

\end{frame}
%%%%%%%%%%%%%%%%%%%%%%%%%%%%%%%%%%%%%%%%%%%%%%%%%%%%%%%%%%%%%%%%%%%%%%%%%%%%%%%%

%%%%%%%%%%%%%%%%%%%%%%%%%%%%%%%%%%%%%%%%%%%%%%%%%%%%%%%%%%%%%%%%%%%%%%%%%%%%%%%%
\begin{frame}[fragile]
  \frametitle{Other Tools}


    AMD CodeAnalyst---based on oprofile.\\[1em]

    WAIT
      \begin{itemize}
        \item IBM's tool tells you what operations your JVM is waiting on while
          idle.
        \item Non-free and not available.
      \end{itemize}

\end{frame}
%%%%%%%%%%%%%%%%%%%%%%%%%%%%%%%%%%%%%%%%%%%%%%%%%%%%%%%%%%%%%%%%%%%%%%%%%%%%%%%%

%%%%%%%%%%%%%%%%%%%%%%%%%%%%%%%%%%%%%%%%%%%%%%%%%%%%%%%%%%%%%%%%%%%%%%%%%%%%%%%%
\begin{frame}
  \frametitle{WAIT: Introduction}


Built for production environments.\\[1em]

Specialized for profiling JVMs, uses JVM hooks to analyze idle time.\\[1em]

Sampling-based analysis; infrequent samples (1--2 per minute!)

\end{frame}

%%%%%%%%%%%%%%%%%%%%%%%%%%%%%%%%%%%%%%%%%%%%%%%%%%%%%%%%%%%%%%%%%%%%%%%%%%%%%%%%

%%%%%%%%%%%%%%%%%%%%%%%%%%%%%%%%%%%%%%%%%%%%%%%%%%%%%%%%%%%%%%%%%%%%%%%%%%%%%%%%
\begin{frame}
  \frametitle{WAIT: Operation}


  At each sample: records each thread's state,
\begin{itemize}
\item call stack;
\item participation in system locks.
\end{itemize}

  Enables WAIT to compute a ``wait state'' (using expert-written rules): \\
what the process is currently doing or waiting on, e.g.
\begin{itemize}
\item disk;
\item GC;
\item network; 
\item blocked; 
\item etc.
\end{itemize}


\end{frame}

%%%%%%%%%%%%%%%%%%%%%%%%%%%%%%%%%%%%%%%%%%%%%%%%%%%%%%%%%%%%%%%%%%%%%%%%%%%%%%%%

%%%%%%%%%%%%%%%%%%%%%%%%%%%%%%%%%%%%%%%%%%%%%%%%%%%%%%%%%%%%%%%%%%%%%%%%%%%%%%%%
\begin{frame}
  \frametitle{WAIT: Workflow}


You:
\begin{itemize}
\item run your application;
\item collect data (using a script or manually); and 
\item upload the data to the server.
\end{itemize}
Server provides
a report.\\
\begin{itemize}
\item You fix the performance problems.\\[1em]
\end{itemize}

Report indicates processor utilization (idle, your application, GC, 
etc); runnable threads; waiting threads (and why they are waiting); 
thread states; and a stack viewer.\\[1em]

Paper presents 6 case studies where WAIT identified performance
problems: deadlocks, server underloads, memory leaks, database
bottlenecks, and excess filesystem activity.


\end{frame}
%%%%%%%%%%%%%%%%%%%%%%%%%%%%%%%%%%%%%%%%%%%%%%%%%%%%%%%%%%%%%%%%%%%%%%%%%%%%%%%%


%%%%%%%%%%%%%%%%%%%%%%%%%%%%%%%%%%%%%%%%%%%%%%%%%%%%%%%%%%%%%%%%%%%%%%%%%%%%%%%%
\begin{frame}[fragile]
  \frametitle{Other Profiling Tools}

  
    Profiling: Not limited to C/C++, or even code.\\[1em]

    You can profile Python using {\tt cProfile}; standard profiling technology.\\[1em]

    Google's Page Speed Tool: profiling for web pages---how can you make your page faster?\\
\begin{itemize}
\item reducing number of DNS lookups;
\item leveraging browser caching;
\item combining images;
\item plus, traditional JavaScript profiling.
\end{itemize}
  
\end{frame}
%%%%%%%%%%%%%%%%%%%%%%%%%%%%%%%%%%%%%%%%%%%%%%%%%%%%%%%%%%%%%%%%%%%%%%%%%%%%%%%%


\end{document}

