
\documentclass[letterpaper,hide notes,xcolor={table,svgnames},pdftex,10pt]{beamer}
\def\showexamples{t}


%\usepackage[svgnames]{xcolor}

%% Demo talk
%\documentclass[letterpaper,notes=show]{beamer}

\usecolortheme{crane}
\setbeamertemplate{navigation symbols}{}

\usetheme{MyPittsburgh}
%\usetheme{Frankfurt}

%\usepackage{tipa}

\usepackage{hyperref}
\usepackage{graphicx,xspace}
\usepackage[normalem]{ulem}
\usepackage{multicol}

\newcommand\SF[1]{$\bigstar$\footnote{SF: #1}}

\usepackage[default]{sourcesanspro}
\usepackage[T1]{fontenc}

\newcounter{tmpnumSlide}
\newcounter{tmpnumNote}

% old question code
%\newcommand\question[1]{{$\bigstar$ \small \onlySlide{2}{#1}}}
% \newcommand\nquestion[1]{\ifdefined \presentationonly \textcircled{?} \fi \note{\par{\Large \textbf{?}} #1}}
% \newcommand\nanswer[1]{\note{\par{\Large \textbf{A}} #1}}


 \newcommand\mnote[1]{%
   \addtocounter{tmpnumSlide}{1}
   \ifdefined\showcues {~\tiny\fbox{\arabic{tmpnumSlide}}}\fi
   \note{\setlength{\parskip}{1ex}\addtocounter{tmpnumNote}{1}\textbf{\Large \arabic{tmpnumNote}:} {#1\par}}}

\newcommand\mmnote[1]{\note{\setlength{\parskip}{1ex}#1\par}}

%\newcommand\mnote[2][]{\ifdefined\handoutwithnotes {~\tiny\fbox{#1}}\fi
% \note{\setlength{\parskip}{1ex}\textbf{\Large #1:} #2\par}}

%\newcommand\mnote[2][]{{\tiny\fbox{#1}} \note{\setlength{\parskip}{1ex}\textbf{\Large #1:} #2\par}}

\newcommand\mquestion[2]{{~\color{red}\fbox{?}}\note{\setlength{\parskip}{1ex}\par{\Large \textbf{?}} #1} \note{\setlength{\parskip}{1ex}\par{\Large \textbf{A}} #2\par}\ifdefined \presentationonly \pause \fi}

\newcommand\blackboard[1]{%
\ifdefined   \showblackboard
  {#1}
  \else {\begin{center} \fbox{\colorbox{blue!30}{%
         \begin{minipage}{.95\linewidth}%
           \hspace{\stretch{1}} Some space intentionally left blank; done at the blackboard.%
         \end{minipage}}}\end{center}}%
         \fi%
}



%\newcommand\q{\tikz \node[thick,color=black,shape=circle]{?};}
%\newcommand\q{\ifdefined \presentationonly \textcircled{?} \fi}

\usepackage{listings}
\lstset{%
  keywordstyle=\bfseries,
  aboveskip=15pt,
  belowskip=15pt,
  captionpos=b,
  identifierstyle=\ttfamily,
  escapeinside={(*@}{@*)},
  stringstyle=\ttfamiliy,
  frame=lines,
  numbers=left, basicstyle=\scriptsize, numberstyle=\tiny, stepnumber=0, numbersep=2pt}

\usepackage{siunitx}
\newcommand\sius[1]{\num[group-separator = {,}]{#1}\si{\micro\second}}
\newcommand\sims[1]{\num[group-separator = {,}]{#1}\si{\milli\second}}
\newcommand\sins[1]{\num[group-separator = {,}]{#1}\si{\nano\second}}
\sisetup{group-separator = {,}, group-digits = true}

%% -------------------- tikz --------------------
\usepackage{tikz}
\usetikzlibrary{positioning}
\usetikzlibrary{arrows,backgrounds,automata,decorations.shapes,decorations.pathmorphing,decorations.markings,decorations.text}

\tikzstyle{place}=[circle,draw=blue!50,fill=blue!20,thick, inner sep=0pt,minimum size=6mm]
\tikzstyle{transition}=[rectangle,draw=black!50,fill=black!20,thick, inner sep=0pt,minimum size=4mm]

\tikzstyle{block}=[rectangle,draw=black, thick, inner sep=5pt]
\tikzstyle{bullet}=[circle,draw=black, fill=black, thin, inner sep=2pt]

\tikzstyle{pre}=[<-,shorten <=1pt,>=stealth',semithick]
\tikzstyle{post}=[->,shorten >=1pt,>=stealth',semithick]
\tikzstyle{bi}=[<->,shorten >=1pt,shorten <=1pt, >=stealth',semithick]

\tikzstyle{mut}=[-,>=stealth',semithick]

\tikzstyle{treereset}=[dashed,->, shorten >=1pt,>=stealth',thin]

\usepackage{ifmtarg}
\usepackage{xifthen}
\makeatletter
% new counter to now which frame it is within the sequence
\newcounter{multiframecounter}
% initialize buffer for previously used frame title
\gdef\lastframetitle{\textit{undefined}}
% new environment for a multi-frame
\newenvironment{multiframe}[1][]{%
\ifthenelse{\isempty{#1}}{%
% if no frame title was set via optional parameter,
% only increase sequence counter by 1
\addtocounter{multiframecounter}{1}%
}{%
% new frame title has been provided, thus
% reset sequence counter to 1 and buffer frame title for later use
\setcounter{multiframecounter}{1}%
\gdef\lastframetitle{#1}%
}%
% start conventional frame environment and
% automatically set frame title followed by sequence counter
\begin{frame}%
\frametitle{\lastframetitle~{\normalfont(\arabic{multiframecounter})}}%
}{%
\end{frame}%
}
\makeatother

\makeatletter
\newdimen\tu@tmpa%
\newdimen\ydiffl%
\newdimen\xdiffl%
\newcommand\ydiff[2]{%
    \coordinate (tmpnamea) at (#1);%
    \coordinate (tmpnameb) at (#2);%
    \pgfextracty{\tu@tmpa}{\pgfpointanchor{tmpnamea}{center}}%
    \pgfextracty{\ydiffl}{\pgfpointanchor{tmpnameb}{center}}%
    \advance\ydiffl by -\tu@tmpa%
}
\newcommand\xdiff[2]{%
    \coordinate (tmpnamea) at (#1);%
    \coordinate (tmpnameb) at (#2);%
    \pgfextractx{\tu@tmpa}{\pgfpointanchor{tmpnamea}{center}}%
    \pgfextractx{\xdiffl}{\pgfpointanchor{tmpnameb}{center}}%
    \advance\xdiffl by -\tu@tmpa%
}
\makeatother
\newcommand{\copyrightbox}[3][r]{%
\begin{tikzpicture}%
\node[inner sep=0pt,minimum size=2em](ciimage){#2};
\usefont{OT1}{phv}{n}{n}\fontsize{4}{4}\selectfont
\ydiff{ciimage.south}{ciimage.north}
\xdiff{ciimage.west}{ciimage.east}
\ifthenelse{\equal{#1}{r}}{%
\node[inner sep=0pt,right=1ex of ciimage.south east,anchor=north west,rotate=90]%
{\raggedleft\color{black!50}\parbox{\the\ydiffl}{\raggedright{}#3}};%
}{%
\ifthenelse{\equal{#1}{l}}{%
\node[inner sep=0pt,right=1ex of ciimage.south west,anchor=south west,rotate=90]%
{\raggedleft\color{black!50}\parbox{\the\ydiffl}{\raggedright{}#3}};%
}{%
\node[inner sep=0pt,below=1ex of ciimage.south west,anchor=north west]%
{\raggedleft\color{black!50}\parbox{\the\xdiffl}{\raggedright{}#3}};%
}
}
\end{tikzpicture}
}


%% --------------------

%\usepackage[excludeor]{everyhook}
%\PushPreHook{par}{\setbox0=\lastbox\llap{MUH}}\box0}

%\vspace*{\stretch{1}

%\setbox0=\lastbox \llap{\textbullet\enskip}\box0}

\setlength{\parskip}{\fill}

\newcommand\noskips{\setlength{\parskip}{1ex}}
\newcommand\doskips{\setlength{\parskip}{\fill}}

\newcommand\xx{\par\vspace*{\stretch{1}}\par}
\newcommand\xxs{\par\vspace*{2ex}\par}
\newcommand\tuple[1]{\langle #1 \rangle}
\newcommand\code[1]{{\sf \footnotesize #1}}
\newcommand\ex[1]{\uline{Example:} \ifdefined \presentationonly \pause \fi
  \ifdefined\showexamples#1\xspace\else{\uline{\hspace*{2cm}}}\fi}

\newcommand\ceil[1]{\lceil #1 \rceil}


\AtBeginSection[]
{
   \begin{frame}
       \frametitle{Outline}
       \tableofcontents[currentsection]
   \end{frame}
}



\pgfdeclarelayer{edgelayer}
\pgfdeclarelayer{nodelayer}
\pgfsetlayers{edgelayer,nodelayer,main}

\tikzstyle{none}=[inner sep=0pt]
\tikzstyle{rn}=[circle,fill=Red,draw=Black,line width=0.8 pt]
\tikzstyle{gn}=[circle,fill=Lime,draw=Black,line width=0.8 pt]
\tikzstyle{yn}=[circle,fill=Yellow,draw=Black,line width=0.8 pt]
\tikzstyle{empty}=[circle,fill=White,draw=Black]
\tikzstyle{bw} = [rectangle, draw, fill=blue!20, 
    text width=4em, text centered, rounded corners, minimum height=2em]
    
    \newcommand{\CcNote}[1]{% longname
	This work is licensed under the \textit{Creative Commons #1 3.0 License}.%
}
\newcommand{\CcImageBy}[1]{%
	\includegraphics[scale=#1]{creative_commons/cc_by_30.pdf}%
}
\newcommand{\CcImageSa}[1]{%
	\includegraphics[scale=#1]{creative_commons/cc_sa_30.pdf}%
}
\newcommand{\CcImageNc}[1]{%
	\includegraphics[scale=#1]{creative_commons/cc_nc_30.pdf}%
}
\newcommand{\CcGroupBySa}[2]{% zoom, gap
	\CcImageBy{#1}\hspace*{#2}\CcImageNc{#1}\hspace*{#2}\CcImageSa{#1}%
}
\newcommand{\CcLongnameByNcSa}{Attribution-NonCommercial-ShareAlike}

\newenvironment{changemargin}[1]{% 
  \begin{list}{}{% 
    \setlength{\topsep}{0pt}% 
    \setlength{\leftmargin}{#1}% 
    \setlength{\rightmargin}{1em}
    \setlength{\listparindent}{\parindent}% 
    \setlength{\itemindent}{\parindent}% 
    \setlength{\parsep}{\parskip}% 
  }% 
  \item[]}{\end{list}} 




\title{Lecture 14 --- OpenMP Tasks }

\author{Patrick Lam \\ \small \texttt{patrick.lam@uwaterloo.ca}}
\institute{Department of Electrical and Computer Engineering \\
  University of Waterloo}
\date{\today}


\begin{document}

\begin{frame}
  \titlepage

 \end{frame}


\begin{frame}[fragile]
\frametitle{Task Manager}

The main new feature in OpenMP 3.0 is the notion of \alert{tasks}. 

When the program executes a \verb+#pragma omp task+ statement, the code
inside the task is split off as a task and scheduled to run sometime
in the future. 

Tasks are more flexible than parallel sections.

They also have lower overhead.

\end{frame}


\begin{frame}
\frametitle{Tasks}

 \begin{center}
    {\tt \#pragma omp }{\bf task} {\it [clause [[,] clause]*]}
  \end{center}~\\

Generates a task for a thread in the team to run.
     When a thread enters the region it may:
\begin{itemize}
        \item immediately execute the task; or
        \item defer its execution. (any other thread may be assigned the task)
\end{itemize}

  Allowed Clauses: {\bf if, final, untied, default, mergeable, private,
  firstprivate, shared}

\end{frame}


\begin{frame}
\frametitle{{\tt if} Clause}

 \begin{center}
  {\bf if} {\it(scalar-logical-expression)}
  \end{center}
 
    When expression is {\tt false}, generates an \alert{undeferred task}.

The generating task region is suspended until the undeferred task finishes.

\begin{center}
\includegraphics[width=0.4\textwidth]{images/doitnow.jpeg}
\end{center}

\end{frame}

\begin{frame}
\frametitle{{\tt final} Clause}

  \begin{center}
  {\bf final} {\it(scalar-logical-expression)}
  \end{center}

    When expression is {\tt true}, generates a final task.
    
    All tasks within a final task are {\it included}.
    
    Included tasks are undeferred and also execute immediately in the same thread.

\end{frame}


\begin{frame}[fragile]
\frametitle{Examples of if and final}

  \begin{lstlisting}[language=C]
void foo () {
    int i;
    #pragma omp task if(0) // This task is undeferred
    {
        #pragma omp task
        // This task is a regular task
        for (i = 0; i < 3; i++) {
            #pragma omp task
            // This task is a regular task
            bar();
        }
    }
    #pragma omp task final(1) // This task is a regular task
    {
        #pragma omp task // This task is included
        for (i = 0; i < 3; i++) {
            #pragma omp task
            // This task is also included
            bar();
        }
    }
}
  \end{lstlisting}

\end{frame}


\begin{frame}
\frametitle{Untied we stan... wait...}

\begin{center}
  {\bf untied}
\end{center}
  \begin{itemize}
    \item A suspended task can be resumed by any thread.
    \item ``untied'' is ignored if used with {\bf final}.
    \item Interacts poorly with thread-private variables and {\tt gettid()}.
  \end{itemize}


\end{frame}



\begin{frame}
\frametitle{Mergeable}

\begin{center}
  {\bf mergeable}
\end{center}

  \begin{itemize}
    \item For an undeferred or included task,
    allows the implementation to generate a merged task instead.
    \item In a merged task, the implementation may re-use the environment from its generating task (as if there was no task directive).
  \end{itemize}

\begin{center}
	\includegraphics[width=0.4\textwidth]{images/mindmeld.jpg}
\end{center}

\end{frame}


\begin{frame}[fragile]
\frametitle{Bad Mergeable Example}

  \begin{lstlisting}[language=C]
#include <stdio.h>
void foo () {
    int x = 2;
    #pragma omp task mergeable
    {
        x++; // x is by default firstprivate
    }
    #pragma omp taskwait
    printf("%d\n",x); // prints 2 or 3
}
  \end{lstlisting}
  
    This is an incorrect usage of {\bf mergeable}: the output depends
      on whether or not the task got merged.
  
  
    Merging tasks (when safe) produces more efficient code.
\end{frame}


\begin{frame}[fragile]
\frametitle{Yield}

  \begin{center}
    {\tt \#pragma omp }{\bf taskyield}
  \end{center}

This directive specifies that the current task can be suspended in favour of another task.

  Here's a good use of {\bf taskyield}.

  \begin{lstlisting}[language=C]
void foo (omp_lock_t * lock, int n) {
    int i;
    for ( i = 0; i < n; i++ )
    #pragma omp task
    {
        something_useful();
        while (!omp_test_lock(lock)) {
            #pragma omp taskyield
        }
        something_critical();
        omp_unset_lock(lock);
    }
}
  \end{lstlisting}



\end{frame}



\begin{frame}
\frametitle{Taskwait}

  \begin{center}
    {\tt \#pragma omp }{\bf taskwait}
  \end{center}~\\[1em]

     Waits for the completion of the current task's child tasks.

\begin{center}
\includegraphics[width=0.6\textwidth]{images/stillwaiting.jpg}
\end{center}

\end{frame}


\begin{frame}[fragile]
\frametitle{Task Example 1: Web Server}

\begin{lstlisting}[language=C,morekeywords={foreach,pragma,omp,parallel,single,nowait,task,untied,barrier,taskyield}]
#pragma omp parallel
  /* a single thread manages the connections */
  #pragma omp single nowait
  while (!end) {
    process any signals
    foreach request from the blocked queue {
      if (request dependencies are met) {
        extract from the blocked queue
        /* create a task for the request */
        #pragma omp task untied
          serve_request(request);
      }
    }
    if (new connection) {
      accept_connection();
      /* create a task for the request */
      #pragma omp task untied
        serve_request(new connection);
    }
    select();
  }
\end{lstlisting}

\end{frame}


\begin{frame}[fragile]
\frametitle{Task Example 2: Tree Traversal}

  \begin{lstlisting}[language=C]
struct node {
    struct node *left;
    struct node *right;
};
extern void process(struct node *);

void traverse(struct node *p) {
    if (p->left) {
        #pragma omp task
        // p is firstprivate by default
        traverse(p->left);
    }
    if (p->right) {
        #pragma omp task
        // p is firstprivate by default
        traverse(p->right);
    }
    process(p);
}    
  \end{lstlisting}

To guarantee a post-order traversal,
insert an explicit \verb+#pragma omp taskwait+
after the two calls to {\tt traverse} and before the
call to {\tt process}.

\end{frame}


\begin{frame}[fragile]
\frametitle{Task Example 3: Linked List Processing}
 \begin{lstlisting}[language=C]
// node struct with data and pointer to next
extern void process(node* p);

void increment_list_items(node* head) {
    #pragma omp parallel
    {
        #pragma omp single
        {
            node * p = head;
            while (p) {
                #pragma omp task
                {
                    process(p);
                }
                p = p->next;
            }
        }
    }
}
  \end{lstlisting}

\end{frame}


\begin{frame}[fragile]
\frametitle{Make ALL the Tasks!}

Let's see what happens
if we spawn lots of tasks in a {\tt single} directive.

  \begin{lstlisting}[language=C]
#define LARGE_NUMBER 10000000
double item[LARGE_NUMBER];
extern void process(double);

int main() {
    #pragma omp parallel
    {
        #pragma omp single
        {
            int i;
            for (i=0; i<LARGE_NUMBER; i++) {
                #pragma omp task
                // i is firstprivate, item is shared
                process(item[i]);
            }
        }
    }
}
  \end{lstlisting}

\end{frame}


\begin{frame}[fragile]
\frametitle{Make ALL the Tasks!}

In this case, the main loop (which executes in one thread only, due to {\tt single}) generates tasks and queues them for execution.

When too many tasks get generated and are waiting, OpenMP suspends the main thread, runs some tasks, then resumes the loop in the main thread.

Any thread may pick up a task and execute it. Without {\tt untied}, a thread that starts a task has to finish running that task.


\end{frame}

\begin{frame}
\frametitle{Over to You!}


If we {\tt untied} the spawned tasks, that would enable the tasks to
migrate between threads when suspended.


\begin{center}
	\includegraphics[width=0.5\textwidth]{images/baton.jpg}
\end{center}


Avoid threadprivate data that
will be wrong after a thread migration.

\end{frame}


\begin{frame}[fragile]
\frametitle{More Scoping}

Besides the {\tt shared}, {\tt private} and {\tt threadprivate}, OpenMP also 
supports {\tt firstprivate} and {\tt lastprivate},

Firstprivate:
  \begin{lstlisting}
int x;

void* run(void* arg) {
    int thread_x = x;
    // use thread_x
}
  \end{lstlisting}


Lastprivate:
  \begin{lstlisting}
int x;

void* run(void* arg) {
    int thread_x;
    // use thread_x
    if (last_iteration) {
        x = thread_x;
    }
}
  \end{lstlisting}

\end{frame}


\begin{frame}
\frametitle{Copy-In}

{\tt copyin} is like firstprivate, but for threadprivate variables.


\begin{center}
	\includegraphics[width=0.4\textwidth]{images/copying.png}
\end{center}


\end{frame}

\begin{frame}[fragile]
\frametitle{Copy-In}

Pseudocode for {\tt copyin}:
  \begin{lstlisting}[language=C]
int x;
int x[NUM_THREADS];

void* run(void* arg) {
  x[thread_num] = x;
  // use x[thread_num]
}
  \end{lstlisting}

The {\tt copyprivate} clause is only used with {\tt single}.


It copies the specified private variables from the thread to all other
threads. It cannot be used with {\tt nowait}.

\end{frame}



\begin{frame}[fragile]
\frametitle{Catch Me Outside...}

\begin{lstlisting}[language=C,morekeywords={foreach,pragma,omp,parallel,single,nowait,task,untied,barrier,taskyield}]
int tid, a, b;

#pragma omp threadprivate(a)

int main(int argc, char *argv[])
{
    printf("Parallel #1 Start\n");
    #pragma omp parallel private(b, tid)
    {
        tid = omp_get_thread_num();
        a = tid;
        b = tid;
        printf("T%d: a=%d, b=%d\n", tid, a, b);
    }

    printf("Sequential code\n");
    printf("Parallel #2 Start\n");
    #pragma omp parallel private(tid)
    {
        tid = omp_get_thread_num();
        printf("T%d: a=%d, b=%d\n", tid, a, b);
    }

    return 0;
}    
  \end{lstlisting}

\end{frame}


\begin{frame}[fragile]
\frametitle{Catch Me Outside...}

Produces as output: 
\begin{lstlisting}[language=C,morekeywords={foreach,pragma,omp,parallel,single,nowait,task,untied,barrier,taskyield}]
% ./a.out
Parallel #1 Start
T6: a=6, b=6
T1: a=1, b=1
T0: a=0, b=0
T4: a=4, b=4
T2: a=2, b=2
T3: a=3, b=3
T5: a=5, b=5
T7: a=7, b=7
Sequential code
Parallel #2 Start
T0: a=0, b=0
T6: a=6, b=0
T1: a=1, b=0
T2: a=2, b=0
T5: a=5, b=0
T7: a=7, b=0
T3: a=3, b=0
T4: a=4, b=0
\end{lstlisting}



\end{frame}



\end{document}

