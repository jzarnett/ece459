\documentclass[letterpaper,10pt]{article}

\usepackage{titling}
\usepackage{listings}
\usepackage{url}
\usepackage{setspace}
\usepackage{subfig}
\usepackage{sectsty}
\usepackage{pdfpages}
\usepackage{colortbl}
\usepackage{multirow}
\usepackage{multicol}
\usepackage{relsize}
\usepackage{amsmath}
\usepackage{fancyvrb}
\usepackage{amsmath,amssymb,amsthm,graphicx,xspace}
\usepackage[titlenotnumbered,noend,noline]{algorithm2e}
\usepackage[compact]{titlesec}
\usepackage{XCharter}
\usepackage[T1]{fontenc}
\usepackage{tikz}
\usetikzlibrary{arrows,automata,shapes,trees,matrix,chains,scopes,positioning,calc}
\tikzstyle{block} = [rectangle, draw, fill=blue!20, 
    text width=2.5em, text centered, rounded corners, minimum height=2em]
\tikzstyle{bw} = [rectangle, draw, fill=blue!20, 
    text width=4em, text centered, rounded corners, minimum height=2em]


\let\LaTeXtitle\title
\renewcommand{\title}[1]{\LaTeXtitle{\textsf{#1}}}


\addtolength{\oddsidemargin}{-1.000in}
\addtolength{\evensidemargin}{-0.500in}
\addtolength{\textwidth}{2.0in}
\addtolength{\topmargin}{-1.000in}
\addtolength{\textheight}{1.75in}
\addtolength{\parskip}{\baselineskip}
\setlength{\parindent}{0in}
\renewcommand{\baselinestretch}{1.5}

\singlespace

\date{}

\begin{document}
\title{\bf\LARGE ECE~459: Programming for Performance\\ (Winter 2016)}
\author{Department of Electrical \& Computer Engineering \\
                University of Waterloo}
\renewcommand{\today}{}
\maketitle
\vspace*{-2em}

\section*{About the Course}

\paragraph{Undergraduate Calendar Description} ``Profiling computer systems; bottlenecks, Amdahl's law. Concurrency: threads and locks. Techniques for programming multicore processors; cache consistency. Transactional memory. Streaming architectures, vectorization, and SIMD. High-performance programming languages.''

\paragraph{Prerequisites} ECE~254 or SE~350; Level at least 4A Electrical Engineering or Computer Engineering or Software Engineering.

Or, to put that in less formal terms: remember when we talked about semaphores and mutexes and all of that? If not, well, \url{https://github.com/jzarnett/ece254/tree/master/lectures}. Also, I really hope you feel comfortable with programming. This isn't the place for you if you're not feeling confident in programming. Really. 

\section*{Brief Overview}

Many modern software systems must process large amounts of data, either in the
form of huge data sets or vast numbers of (concurrent) transactions.  This
course introduces students to techniques for profiling, rearchitecting, and
implementing software systems that can handle industrial-sized
inputs. These techniques will enable you to design and build
critical software infrastructure, especially in an age of Big Data.

While you may have seen some of these ideas in the context of
operating systems (ECE354/SE350/CS350) and concurrency (CS343), this course gives you tools to make code run
faster. The focus in OS is understanding and implementing the
primitives; our focus is on using them effectively. 

We will sometimes see implementation details that you need to get right to write
certain applications, but as with any university-level course, this course
focusses more on the concepts than magic invocations, so that you can continue
to apply the basic ideas after the technologies inevitably change.


\section*{General Information}

\paragraph{Course Website} As is standard, information will be posted on Learn. But you will see all the magic at \url{https://github.com/jzarnett/ece459}. The primary source for course materials is github.

\noindent {\bf Schedule:}
\\

\noindent
\hspace*{2em} \begin{tabular}{l|l|l|l}
  Lectures & MF & 8:30 -- 9:50 & EIT~1015\\ \hline
  Tutorial & F & 14:30-15:20 & RCH~301\\
  \end{tabular}
  
  Midterm and final exams to be announced.

Reading week will be 15-19 February. 25 March is a holiday (Good Friday), so we will follow the Friday schedule on Monday 4 April. 

\textit{Final Exam}: The final exam dates are 8 April - 23 April. Partway through the term, the registrar's office will announce the exact date, time, and location of the exam. It could fall at any time in this period. Note that student travel plans are \emph{not} considered an acceptable reason for missing an exam. When it is announced, please alert your instructor immediately if you have a conflict.

\paragraph{About Prof. Zarnett.}
I graduated from the Computer Engineering program at Waterloo (under the previous curriculum), and have since earned my Master's Degree (also at Waterloo) and my P.Eng. license. For the last 2+ years I have also been teaching here at UW and the other courses I have taught and worked on include ECE~150, ECE~155, ECE~254, ECE~290, and MTE~241.

In addition to being your instructor, I work full time in software engineering in industry. It's my job to help you learn, so contact me if you feel the need. The vast majority of the week, I will not be on campus, so stopping by my office is a very inefficient way to find me. I'm more than happy to answer questions by e-mail, and if they can't be answered by e-mail, we can set up an appointment for office hours.

\paragraph{About the Teaching Assistants.}

Teaching assistants can help you with the course material, including tutorials, labs, and exams. They will be present in the labs and will conduct the tutorials. TA Office hours will be posted in Learn as calendar events, but you can also make an appointment.


\noindent {\bf Teaching Assistants:} \\

\noindent
\begin{tabular}{ll}
\hspace*{2em} \begin{minipage}{.4\textwidth}
Xi Cheng\\
Email: {\tt x22cheng@uwaterloo.ca}

\end{minipage} &
\hspace*{2em} \begin{minipage}{.4\textwidth}
Morteza Nabavi\\
Email: {\tt mnabavi@uwaterloo.ca}

\end{minipage} \\[3em]
\hspace*{2em} \begin{minipage}{.4\textwidth}
Husam Suleiman\\
Email: {\tt hsuleima@uwaterloo.ca}

\end{minipage} &



\end{tabular}


\section*{Course Description}

\paragraph{Topics.} Here is a detailed list of topics and estimated lecture
hours for each topic. 

\section*{Reference Material}
You might find this book useful.

\begin{quote}
    Software Performance and Scalability: A Quantitative Approach, Henry H. Liu, John Wiley \& Sons, 2009. 
\end{quote}


\section*{Evaluation}
This course includes assignments, a midterm, and a final
examination. Too many fourth-year courses have projects, so I'm not
going to pile on another project here.\vspace*{1em}

\begin{tabular}{l@{\hspace*{5em}}l}
Assignments & 60\% (4 at 15\% each) \\
Midterm & 10\% \\
Final exam & 30\% \\
\end{tabular}

A grade of INC (Incomplete) is to be assigned if you do not attempt all assignments.  If the assignments are not completed within 4 months of the end of the course to the satisfaction of the course instructor, the INC (Incomplete) grade will be converted to FTC  (Failure to Complete).  If the assignments are completed successfully  within  the  required  timeframe,  the  overall  grade  will  be  calculated using the  rules  for  late assignment submission. Furthermore, any submitted assignment must be an honest attempt - submitting an empty project or similar is not adequate. 

The University rules say if you miss the final exam, without an acceptable reason, your grade in the class will be DNW - Did Not Write. This is very undesirable. Show up for the final exam.

\paragraph{Assignments.} Since this course has ``programming'' in the 
title, you will be expected to write code for these assignments.  Here
is a projected list of assignments for this course. I plan to have 4
assignments.
\begin{enumerate}
\item implementing manual parallelization for servers with Pthreads and C++11 asynchronous I/O;
\item using compiler-provided automatic parallelization and OpenMP;
\item identifying opportunities for optimization (e.g. parallelization) and implementing them; and
\item GPU programming with OpenCL.
\end{enumerate}
Assignment handin will be done via git.

There's a chance I will replace the 4th one. I'll let you know. You'll have two weeks to do each assignment.

\paragraph{Exams.} Exams will be open-book, open-notes, no electronics, no communication devices.

\paragraph{Group work.} 
You may discuss assignments with others, but I expect each of you to
do each assignment independently. Acceptable collaboration includes
discussing ideas and structures with others, as well as helping others
debug their code. If your code is too close in structure to someone
else's code, you are going to have a problem. The best way to avoid
such problems is by (1) not sending your code around; and (2) not
writing down anything beyond general notes (pseudocode) about other
peoples' code. I will follow UW's Policy 71 if I discover any cases of
plagiarism (and I have). 

I want to emphasize that I take the issue of plagiarism very seriously, and so does the University of Waterloo. If you are uncertain about this subject, please seek some guidance. There are many resources available to you. You can check the university policies, talk to the course instructor ECE undergrad office, et cetera.

Or, let's sum this up in two short instructions:
\begin{enumerate}
	\item Acknowledge the work of others. 
	\item If you are uncertain, ask!
\end{enumerate}


\paragraph{Lateness.} Also known as ``Grace Days''. You have 4 days of lateness to use on 
submissions throughout the term. Each day you hand in something late
consumes one of the days of lateness. The fifth day of lateness causes
your lowest assignment mark to be halved, while the sixth day causes
both assignment marks to be halved. If you hand in something and you
have more than 6 days of lateness, I'll start converting marks to 0
and dropping the associated late days. You don't
get any credit for unused late days.

For example, you may hand in A1 one day late, A2 two days late, A3 1
one day late, and everything else on time.  Or you can hand A2 four
days late, if you hand in everything else on time. Finally, if you
hand in A1 3 days late, A2 1 day late, A3 3 days late, and
everything else on time, I will either give you a 0 for A1, leaving you
with 4 late days, or give you a 0 for A2, leaving you with 5 late days
and causing your mark for A1 to be halved. I'll choose the option
which gives you more marks.

\paragraph{Re-marking}
If you believe that your grade on an a written, submitted deliverable (e.g., a midterm exam question) is incorrect or unfair, you may ask that it be re-marked. Note that lab demonstrations cannot be re-marked. To request that a question be re-marked, you will need to submit your request on a sheet of paper, in writing, to me (the instructor). You may submit it to me in person, or ask an administrative assistant at the Electrical \& Computer Engineering undergraduate office to put it in my mailbox. Please do not hang around outside my office hoping to find me (this is really inefficient). Please do not submit it to a TA or lab staff. 

When you submit your request, it should include the following: (1) Your name and student ID number; (2) a clear indication of which question or part of the deliverable is to be re-marked; and (3) an explanation of why you believe the grade assigned was incorrect.

If you received the marked version of the deliverable back (e.g., the midterm exam), please submit that alongside your written request. Staple them (not paperclip, not some sort of origami) together so they do not get separated.

I will accept items for re-marking any time before the final exam, including as I'm arriving to the room where the final exam is to be written (I have actually done this myself as a student). Be forewarned, when a deliverable is being re-marked, your grade could go up, it could stay the same, or it could go down. I will notify you of the outcome and attempt to return the deliverable you submitted (if any).

\paragraph{Extra Credit}
In this class, there will be no opportunities to earn extra credit. Make-up assignments, labs, or examinations will not be offered under any circumstances.

\paragraph{Attendance \& Illness}

My personal opinion on attending classes is that it is usually a good idea to attend lectures. That said, this is university and you are capable of deciding for yourself if you are going to the lecture or not. Attendance is not taken and not graded. No attendance is taken in tutorials or labs, but at least one member of your lab group will need to be present in the labs to present lab deliverables.

Some advice Professor Gebotys gave my class when I was a student: If you are tired, go sleep at home. Sleeping in the lecture doesn't work; you will get poor quality of sleep and you won't learn the material while you're asleep, either.

During the term, you may need arrive late to a class or leave partway through, because of co-op interviews. This is not a problem, as long as you are not disruptive when arriving/departing.

If you feel ill, you should seek appropriate medical attention. If you miss an exam for health reasons, you need a verification of illness form. Forms can be completed by the physicians at Health Services. If you anticipate missing a deliverable deadline or an examination for a non-medical reason, you should contact me as soon as you are aware of the problem. Given sufficient notice, alternate arrangements may be possible. Alternate arrangements are rare and at my discretion.

\section*{University Policies}

\paragraph{Academic Integrity}
In order to maintain a culture of academic integrity, members of the University of Waterloo community are expected to promote honesty, trust, fairness, respect and responsibility. Check\\
\texttt{www.uwaterloo.ca/academicintegrity/} for more information.

\paragraph{Grievance}
A student who believes that a decision affecting some aspect of his/her university life has been unfair or unreasonable may have grounds for initiating a grievance. Read Policy 70, Student Petitions and Grievances, Section 4, \texttt{adm.uwaterloo.ca/infosec/Policies/policy70.htm} \\
If in doubt, contact the department's administrative assistant, who will provide further assistance.

\paragraph{Discipline}
A student is expected to know what constitutes academic integrity (see above section) to avoid committing an academic offence, and to take responsibility for his/her actions. A student who is unsure whether an action constitutes an offence, or who needs help in learning how to avoid offences (e.g., plagiarism, cheating) or about ''rules'' for group work/collaboration should seek guidance from the course instructor, academic advisor, or the undergraduate Associate Dean. For information on categories of offences and types of penalties, students should refer to Policy 71, Student Discipline, \texttt{www.adm.uwaterloo.ca/infosec/Policies/policy71.htm} . For typical penalties check Guidelines for the Assessment of Penalties, see \\\texttt{www.adm.uwaterloo.ca/infosec/guidelines/penaltyguidelines.htm} .

\paragraph{Appeals}
A decision made or penalty imposed under Policy 70 (Student Petitions and Grievances) (other than a petition) or Policy 71 (Student Discipline) may be appealed if there is a ground. A student who believes he/she has a ground for an appeal should refer to Policy 72 (Student Appeals)\\~\texttt{www.adm.uwaterloo.ca/infosec/Policies/policy72.htm}.

\paragraph{Privacy}
Questions about the collection, use, and disclosure of personal information by the University, should be directed to the Freedom of Information and Privacy Coordinator, Secretariat, University of Waterloo, 200 University Avenue West, Waterloo, Ontario, Canada N2L 3G1. The email address of the Freedom of Information and Privacy Coordinator is \texttt{fippa@uwaterloo.ca}. See also University of Waterloo Policy 19: Access to and Release of Student Information; Information and Privacy.
\\ \texttt{https://uwaterloo.ca/secretariat/policies-procedures-guidelines/policy-19}

\paragraph{Note for Students with Special Needs}
The AccessAbility Services (formerly known as OPD) located in Needles Hall, Room 1132, collaborates with all academic departments to arrange appropriate accommodations for students with disabilities without compromising the academic integrity of the curriculum. If you require academic accommodations to lessen the impact of your disability, please register with the AccessAbility Services office at the beginning of each academic term.
\end{document}
