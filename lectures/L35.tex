\documentclass[letterpaper,10pt]{article}

\usepackage{titling}
\usepackage{listings}
\usepackage{url}
\usepackage{setspace}
\usepackage{subfig}
\usepackage{sectsty}
\usepackage{pdfpages}
\usepackage{colortbl}
\usepackage{multirow}
\usepackage{multicol}
\usepackage{relsize}
\usepackage{amsmath}
\usepackage{fancyvrb}
\usepackage{amsmath,amssymb,amsthm,graphicx,xspace}
\usepackage[titlenotnumbered,noend,noline]{algorithm2e}
\usepackage[compact]{titlesec}
\usepackage{XCharter}
\usepackage[T1]{fontenc}
\usepackage{enumitem}
\usepackage{tikz}
\usetikzlibrary{arrows,automata,shapes,trees,matrix,chains,scopes,positioning,calc}
\tikzstyle{block} = [rectangle, draw, fill=blue!20, 
    text width=2.5em, text centered, rounded corners, minimum height=2em]
\tikzstyle{bw} = [rectangle, draw, fill=blue!20, 
    text width=4em, text centered, rounded corners, minimum height=2em]

\newcommand{\CPP}{C\nolinebreak\hspace{-.05em}\raisebox{.4ex}{\tiny\bf +}\nolinebreak\hspace{-.10em}\raisebox{.4ex}{\tiny\bf +}}
\def\CPP{{C\nolinebreak[4]\hspace{-.05em}\raisebox{.4ex}{\tiny\bf ++}}}

\let\LaTeXtitle\title
\renewcommand{\title}[1]{\LaTeXtitle{\textsf{#1}}}


\addtolength{\oddsidemargin}{-1.000in}
\addtolength{\evensidemargin}{-0.500in}
\addtolength{\textwidth}{2.0in}
\addtolength{\topmargin}{-1.000in}
\addtolength{\textheight}{1.75in}
\addtolength{\parskip}{\baselineskip}
\setlength{\parindent}{0in}
\renewcommand{\baselinestretch}{1.5}

\singlespace


\begin{document}

\lecture{35 --- DevOps for P4P}{\term}{Patrick Lam}

Two topics today: 1) DevOps considerations (think
big); 2) the cost of scalability (think small).

\section*{DevOps for P4P}
% this part took the whole 50 minutes, at least on the blackboard.
So far, we've talked almost exclusively about one-off computations:
you want to figure out the answer to a question, and you write code to
do that. Our assignments have been like that, for instance. But a lot
of the time we want to keep systems running over time. That gets us
into the notion of operations. 

The theme today will be using software development skills in
operations (e.g. system administration, database management, etc).

Even when we've talked about multi-computer tools like MPI and cloud
computing, it still has not been in the context of keeping your
systems operational over longer timescales. The trend today is away
from strict separation between a development team, which writes the
software, and an operations team, which runs the software.

%http://cdn.meme.am/instances/500x/22605665.jpg

Thanks to Chris Jones and Niall Murphy for the following points.

\subsection*{Configuration as code}
Systems have long come with complicated configuration options.
Sendmail is particularly notorious, but apache and nginx aren't super
easy to configure either. The first principle is to treat \emph{configuration as code}.
Therefore:
\begin{itemize}
\item use version control on your configuration.
\item test your configurations: that means that you check that they
  generate expected files, or that they spawn expected
  services. (Behaviours, or outcomes.) Also, configurations should
  ``converge''. Unlike code, they might not terminate; we're talking
  indefinitely-running services, after all. But the CPU usage should
  go down after a while, for instance.
\item aim for a suite of modular services that integrate together smoothly.
\item refactor configuration files (Puppet manifests, Chef recipes, etc);
\item use continuous builds (more on that later).
\end{itemize}

\subsection*{Servers as cattle, not pets}
By servers, I mean servers, or virtual machines, or containers.
At a certain scale (and it's smaller than you think), it's useful to
mass-produce tools for dealing with servers, rather than doing tasks
manually. At a minimum, you need to be able to set up these servers
without manual intervention. They should be able to be spun up 
programmatically.
% http://smithmeadows.com/wp-content/uploads/2012/10/Cattle.jpg

\subsection*{Common infrastructure}
Use APIs to access your infrastructure. Some examples:
\begin{itemize}
\item storage: some sort of access layer to MongoDB or Amazon S3 or whatever;
\item naming and discovery infrastructure (more below);
\item monitoring infrastructure.
\end{itemize}
Try to avoid one-offs by using, for instance, open-source tools when applicable.
Be prepared to build your own tools if needed.

\subsection*{Design for 10$\times$ growth, redesign before 100$\times$}
[original credit: Jeff Dean at Google] This discussion is based on
Martin Fowler's piece on sacrificial architecture:
\url{http://martinfowler.com/bliki/SacrificialArchitecture.html}.

Consider eBay: in 1995, perl scripts; in 1997, C++/Windows; in 2002,
Java.  Each of these architectures was appropriate at the time, but
not as the requirements change. The more sophisticated successor
architectures, however, would have been overkill at an earlier
time. And it's hard to predict what would be needed in the future.
% look up eBay growth stats
%http://martinfowler.com/bliki/images/sacrificialArchitecture/sketch.png
%http://web.archive.org/web/20000510004517/http://www.ebay.com/
% http://www.ebay.com/

\begin{quote}
``Perf is a feature''.\\
\hfill --- Jeff Atwood
\end{quote}
That is, you apply developer time to perf, and you make engineering tradeoffs
to get it. Some thoughts:
\begin{itemize}
\item design with the eventual replacement in mind;
\item don't abandon internal quality (e.g. modularity);
\item sacrifice individual modules at a time, not the whole system;
\item you can also implement new features with a rough draft and deploy to a test audience.
\end{itemize}

\subsection*{Naming}
Naming is one of the hard problems in computer science. There are a
lot of ways to name things. We'll talk about
systems/VMs\footnote{\url{http://mnx.io/blog/a-proper-naming-scheme}},
but naming is necessary for resources of all kinds.

In brief:
\begin{itemize}
\item use canonical one-word names for servers;
\item but, use aliases to specify functions, e.g. 1) geography (nyc); 2) environment (dev/tst/stg/prod); 
3) purpose (app/sql/etc); and 4) serial number.
\end{itemize}
This enables you to have a way of referring to each machine in an absolute sense, but also 
allows you to use functional names when creating dependencies between systems.

\subsection*{Other Topics}
Beyond the five principles above, there are a couple more techniques that particularly apply to
DevOps:

\paragraph{Continuous Integration.} 
This is now a best practice. It's enabled by the use of version control, good tests, and scripted deployments.
It works like this:
\begin{itemize}
\item pull code from version control;
\item build;
\item run tests;
\item report results.
\end{itemize}
What's also key is a social convention to not break the build. 

%https://jenkins-ci.org/sites/default/files/jenkins_logo.png

CI is good for all code, but it's especially good for configuration-as-code, which is especially likely
to break in different environments.

\paragraph{Canarying.}
%http://www.post-gazette.com/image/2013/10/29/ca27,76,1566,1822/Canary.jpg
Deploy new software incrementally alongside production software, also known as ``test in prod''. Sometimes
you just don't know how code is really going to work until you try it. After, of course, you use your best
efforts to make sure the code is good. Steps:
\begin{itemize}
\item stage for deployment;
\item remove canary servers from service;
\item upgrade canary servers;
\item run automatic tests on upgraded canaries;
\item reintroduce canary servers into service;
\item see how it goes!
\end{itemize}
Of course, you should implement your system so that rollback is possible.

\paragraph{Monitoring.}
Here's one way to think about it. 
% track down the source here: Niall Murphy interview.
\begin{itemize}
\item {\bf Alerts}: a human must take action now;
\item {\bf Tickets}: a human must take action soon (hours or days);
\item {\bf Logging}: no need to look at this except for forensic/diagnostic purposes.
\end{itemize}
A common bad situation is logs-as-tickets: you should never be in the
situation where you routinely have to look through logs to find
errors. Write code to scan logs.

\section*{Clusters versus Laptops}
There is a paper about this:

\noindent
Frank McSherry, Michael Isard, Derek G. Murray. ``Scalability! But at what COST?'' HotOS XV.

This part of the lecture is based on the companion blog post\footnote{\url{http://www.frankmcsherry.org/graph/scalability/cost/2015/01/15/COST.html}}.

The key idea: scaling to big data systems introduces substantial overhead. Let's just see how, say, a
laptop compares, in absolute times, to 128-core big data systems.

\paragraph{Conclusion.} Big data systems haven't yet been shown to be obviously good; current evaluation is lacking.
The important metric is not just scalability; absolute
performance matters a lot too.

\paragraph{Methodology.} We'll compare a competent single-threaded implementation to top
big data systems, as described in an OSDI 2014 (top OS conference) paper on GraphX\footnote{\url{https://www.usenix.org/system/files/conference/osdi14/osdi14-paper-gonzalez.pdf}}. The domain: graph processing
algorithms, namely PageRank and graph connectivity (for which the bottleneck is label propagation). The subjects: graphs with billions of edges, amounting to a few
GB of data.

\paragraph{Results.} 128 cores don't consistently beat a laptop at PageRank: e.g. 249--857s on the twitter\_rv dataset for the big data system vs 300s for the laptop, and they are 2$\times$ slower for label
propagation, at 251--1784s for the big data system vs 153s on
twitter\_rv. (See the blog post for the full results).

\paragraph{Wait, there's more.} I keep on saying that we can improve algorithms for additional performance boosts too.
But that doesn't generalize, so it's hard to teach. In this case, two improvements are: using Hilbert curves
for data layout, improving memory locality, which helps a lot for PageRank; and using a union-find algorithm 
(which is also parallelizable). ``10$\times$ faster, 100$\times$ less embarrassing''.  We observe an overall
$2\times$ speedup for PageRank and $10\times$ speedup for label propagation.

\paragraph{Takeaways.} Some thoughts to keep in mind, from the authors:
\begin{itemize}
\item    ``If you are going to use a big data system for yourself, see if it is faster than your laptop.''
\item    ``If you are going to build a big data system for others, see that it is faster than my laptop.''
\end{itemize}



\bibliographystyle{alpha}
\bibliography{459}


\end{document}
