
\documentclass[letterpaper,hide notes,xcolor={table,svgnames},pdftex,10pt]{beamer}
\def\showexamples{t}


%\usepackage[svgnames]{xcolor}

%% Demo talk
%\documentclass[letterpaper,notes=show]{beamer}

\usecolortheme{crane}
\setbeamertemplate{navigation symbols}{}

\usetheme{MyPittsburgh}
%\usetheme{Frankfurt}

%\usepackage{tipa}

\usepackage{hyperref}
\usepackage{graphicx,xspace}
\usepackage[normalem]{ulem}
\usepackage{multicol}

\newcommand\SF[1]{$\bigstar$\footnote{SF: #1}}

\usepackage[default]{sourcesanspro}
\usepackage[T1]{fontenc}

\newcounter{tmpnumSlide}
\newcounter{tmpnumNote}

% old question code
%\newcommand\question[1]{{$\bigstar$ \small \onlySlide{2}{#1}}}
% \newcommand\nquestion[1]{\ifdefined \presentationonly \textcircled{?} \fi \note{\par{\Large \textbf{?}} #1}}
% \newcommand\nanswer[1]{\note{\par{\Large \textbf{A}} #1}}


 \newcommand\mnote[1]{%
   \addtocounter{tmpnumSlide}{1}
   \ifdefined\showcues {~\tiny\fbox{\arabic{tmpnumSlide}}}\fi
   \note{\setlength{\parskip}{1ex}\addtocounter{tmpnumNote}{1}\textbf{\Large \arabic{tmpnumNote}:} {#1\par}}}

\newcommand\mmnote[1]{\note{\setlength{\parskip}{1ex}#1\par}}

%\newcommand\mnote[2][]{\ifdefined\handoutwithnotes {~\tiny\fbox{#1}}\fi
% \note{\setlength{\parskip}{1ex}\textbf{\Large #1:} #2\par}}

%\newcommand\mnote[2][]{{\tiny\fbox{#1}} \note{\setlength{\parskip}{1ex}\textbf{\Large #1:} #2\par}}

\newcommand\mquestion[2]{{~\color{red}\fbox{?}}\note{\setlength{\parskip}{1ex}\par{\Large \textbf{?}} #1} \note{\setlength{\parskip}{1ex}\par{\Large \textbf{A}} #2\par}\ifdefined \presentationonly \pause \fi}

\newcommand\blackboard[1]{%
\ifdefined   \showblackboard
  {#1}
  \else {\begin{center} \fbox{\colorbox{blue!30}{%
         \begin{minipage}{.95\linewidth}%
           \hspace{\stretch{1}} Some space intentionally left blank; done at the blackboard.%
         \end{minipage}}}\end{center}}%
         \fi%
}



%\newcommand\q{\tikz \node[thick,color=black,shape=circle]{?};}
%\newcommand\q{\ifdefined \presentationonly \textcircled{?} \fi}

\usepackage{listings}
\lstset{%
  keywordstyle=\bfseries,
  aboveskip=15pt,
  belowskip=15pt,
  captionpos=b,
  identifierstyle=\ttfamily,
  escapeinside={(*@}{@*)},
  stringstyle=\ttfamiliy,
  frame=lines,
  numbers=left, basicstyle=\scriptsize, numberstyle=\tiny, stepnumber=0, numbersep=2pt}

\usepackage{siunitx}
\newcommand\sius[1]{\num[group-separator = {,}]{#1}\si{\micro\second}}
\newcommand\sims[1]{\num[group-separator = {,}]{#1}\si{\milli\second}}
\newcommand\sins[1]{\num[group-separator = {,}]{#1}\si{\nano\second}}
\sisetup{group-separator = {,}, group-digits = true}

%% -------------------- tikz --------------------
\usepackage{tikz}
\usetikzlibrary{positioning}
\usetikzlibrary{arrows,backgrounds,automata,decorations.shapes,decorations.pathmorphing,decorations.markings,decorations.text}

\tikzstyle{place}=[circle,draw=blue!50,fill=blue!20,thick, inner sep=0pt,minimum size=6mm]
\tikzstyle{transition}=[rectangle,draw=black!50,fill=black!20,thick, inner sep=0pt,minimum size=4mm]

\tikzstyle{block}=[rectangle,draw=black, thick, inner sep=5pt]
\tikzstyle{bullet}=[circle,draw=black, fill=black, thin, inner sep=2pt]

\tikzstyle{pre}=[<-,shorten <=1pt,>=stealth',semithick]
\tikzstyle{post}=[->,shorten >=1pt,>=stealth',semithick]
\tikzstyle{bi}=[<->,shorten >=1pt,shorten <=1pt, >=stealth',semithick]

\tikzstyle{mut}=[-,>=stealth',semithick]

\tikzstyle{treereset}=[dashed,->, shorten >=1pt,>=stealth',thin]

\usepackage{ifmtarg}
\usepackage{xifthen}
\makeatletter
% new counter to now which frame it is within the sequence
\newcounter{multiframecounter}
% initialize buffer for previously used frame title
\gdef\lastframetitle{\textit{undefined}}
% new environment for a multi-frame
\newenvironment{multiframe}[1][]{%
\ifthenelse{\isempty{#1}}{%
% if no frame title was set via optional parameter,
% only increase sequence counter by 1
\addtocounter{multiframecounter}{1}%
}{%
% new frame title has been provided, thus
% reset sequence counter to 1 and buffer frame title for later use
\setcounter{multiframecounter}{1}%
\gdef\lastframetitle{#1}%
}%
% start conventional frame environment and
% automatically set frame title followed by sequence counter
\begin{frame}%
\frametitle{\lastframetitle~{\normalfont(\arabic{multiframecounter})}}%
}{%
\end{frame}%
}
\makeatother

\makeatletter
\newdimen\tu@tmpa%
\newdimen\ydiffl%
\newdimen\xdiffl%
\newcommand\ydiff[2]{%
    \coordinate (tmpnamea) at (#1);%
    \coordinate (tmpnameb) at (#2);%
    \pgfextracty{\tu@tmpa}{\pgfpointanchor{tmpnamea}{center}}%
    \pgfextracty{\ydiffl}{\pgfpointanchor{tmpnameb}{center}}%
    \advance\ydiffl by -\tu@tmpa%
}
\newcommand\xdiff[2]{%
    \coordinate (tmpnamea) at (#1);%
    \coordinate (tmpnameb) at (#2);%
    \pgfextractx{\tu@tmpa}{\pgfpointanchor{tmpnamea}{center}}%
    \pgfextractx{\xdiffl}{\pgfpointanchor{tmpnameb}{center}}%
    \advance\xdiffl by -\tu@tmpa%
}
\makeatother
\newcommand{\copyrightbox}[3][r]{%
\begin{tikzpicture}%
\node[inner sep=0pt,minimum size=2em](ciimage){#2};
\usefont{OT1}{phv}{n}{n}\fontsize{4}{4}\selectfont
\ydiff{ciimage.south}{ciimage.north}
\xdiff{ciimage.west}{ciimage.east}
\ifthenelse{\equal{#1}{r}}{%
\node[inner sep=0pt,right=1ex of ciimage.south east,anchor=north west,rotate=90]%
{\raggedleft\color{black!50}\parbox{\the\ydiffl}{\raggedright{}#3}};%
}{%
\ifthenelse{\equal{#1}{l}}{%
\node[inner sep=0pt,right=1ex of ciimage.south west,anchor=south west,rotate=90]%
{\raggedleft\color{black!50}\parbox{\the\ydiffl}{\raggedright{}#3}};%
}{%
\node[inner sep=0pt,below=1ex of ciimage.south west,anchor=north west]%
{\raggedleft\color{black!50}\parbox{\the\xdiffl}{\raggedright{}#3}};%
}
}
\end{tikzpicture}
}


%% --------------------

%\usepackage[excludeor]{everyhook}
%\PushPreHook{par}{\setbox0=\lastbox\llap{MUH}}\box0}

%\vspace*{\stretch{1}

%\setbox0=\lastbox \llap{\textbullet\enskip}\box0}

\setlength{\parskip}{\fill}

\newcommand\noskips{\setlength{\parskip}{1ex}}
\newcommand\doskips{\setlength{\parskip}{\fill}}

\newcommand\xx{\par\vspace*{\stretch{1}}\par}
\newcommand\xxs{\par\vspace*{2ex}\par}
\newcommand\tuple[1]{\langle #1 \rangle}
\newcommand\code[1]{{\sf \footnotesize #1}}
\newcommand\ex[1]{\uline{Example:} \ifdefined \presentationonly \pause \fi
  \ifdefined\showexamples#1\xspace\else{\uline{\hspace*{2cm}}}\fi}

\newcommand\ceil[1]{\lceil #1 \rceil}


\AtBeginSection[]
{
   \begin{frame}
       \frametitle{Outline}
       \tableofcontents[currentsection]
   \end{frame}
}



\pgfdeclarelayer{edgelayer}
\pgfdeclarelayer{nodelayer}
\pgfsetlayers{edgelayer,nodelayer,main}

\tikzstyle{none}=[inner sep=0pt]
\tikzstyle{rn}=[circle,fill=Red,draw=Black,line width=0.8 pt]
\tikzstyle{gn}=[circle,fill=Lime,draw=Black,line width=0.8 pt]
\tikzstyle{yn}=[circle,fill=Yellow,draw=Black,line width=0.8 pt]
\tikzstyle{empty}=[circle,fill=White,draw=Black]
\tikzstyle{bw} = [rectangle, draw, fill=blue!20, 
    text width=4em, text centered, rounded corners, minimum height=2em]
    
    \newcommand{\CcNote}[1]{% longname
	This work is licensed under the \textit{Creative Commons #1 3.0 License}.%
}
\newcommand{\CcImageBy}[1]{%
	\includegraphics[scale=#1]{creative_commons/cc_by_30.pdf}%
}
\newcommand{\CcImageSa}[1]{%
	\includegraphics[scale=#1]{creative_commons/cc_sa_30.pdf}%
}
\newcommand{\CcImageNc}[1]{%
	\includegraphics[scale=#1]{creative_commons/cc_nc_30.pdf}%
}
\newcommand{\CcGroupBySa}[2]{% zoom, gap
	\CcImageBy{#1}\hspace*{#2}\CcImageNc{#1}\hspace*{#2}\CcImageSa{#1}%
}
\newcommand{\CcLongnameByNcSa}{Attribution-NonCommercial-ShareAlike}

\newenvironment{changemargin}[1]{% 
  \begin{list}{}{% 
    \setlength{\topsep}{0pt}% 
    \setlength{\leftmargin}{#1}% 
    \setlength{\rightmargin}{1em}
    \setlength{\listparindent}{\parindent}% 
    \setlength{\itemindent}{\parindent}% 
    \setlength{\parsep}{\parskip}% 
  }% 
  \item[]}{\end{list}} 




\title{Lecture 32 --- Convergence \& Ergodicity}

\author{Jeff Zarnett\\ \small \texttt{jzarnett@uwaterloo.ca}}
\institute{Department of Electrical and Computer Engineering \\
  University of Waterloo}
\date{\today}


\begin{document}

\begin{frame}
  \titlepage

 \end{frame}


\begin{frame}
\frametitle{Convergence}

An image of what convergence looks like:

\begin{center}
	\includegraphics[width=0.6\textwidth]{images/convergence.png}
\end{center}

We won't concern ourselves with systems where there is no convergence. 

\end{frame}



\begin{frame}
\frametitle{Tim and Enzo}

A small but important digression on the subject of sampling, measurement, and testing.

You have an idea of what an average is, but there are two different types of average---the time average and ensemble average. 


\end{frame}



\begin{frame}
\frametitle{Tim and Enzo Scenario}

Let us just focus on having a single First-Come-First-Serve queue. 

Every second, a new job arrives with probability $p$ and if there is any work to do, the job being worked on is completed with probability $q$ (and $q > p$). 

As a definition, let $N(v)$ equal the number of jobs in the system at a time $v$. 

In the story, Tim and Enzo are trying to simulate the FCFS system to determine what is the average number of jobs in the system.

\end{frame}



\begin{frame}
\frametitle{The Tim Approach}

Tim decides he's going to run it as one really long simulation. 

He simulates the queue over a very long period, logging as he goes, taking a million samples. 

Then he takes the average value over those samples to get the average number of jobs.

\end{frame}



\begin{frame}
\frametitle{The Enzo Approach}

Enzo does something slightly different: instead of having one super long simulation, he does 1000 shorter simulations. 

He waits until the simulation has run for 1000 seconds and then samples the queue at exactly that point, obtaining one value. 

This experiment is restarted with a new random seed. 

So after obtaining a thousand samples, he averages these, and Enzo produces another average number of jobs.


\end{frame}



\begin{frame}
\frametitle{Tim and Enzo}

\begin{center}
	\includegraphics[width=0.8\textwidth]{images/timenzo.png}
\end{center}

So---who has done this correctly, Tim or Enzo?

\end{frame}



\begin{frame}
\frametitle{Time Average vs Ensemble Average}

The time average has potential problems because we are only looking at a single sequence and maybe something very unusual has happened here in this single run. 

The ensemble average is more likely what we talk about when we talk about the system being at ``steady state'' (i.e., past the initial conditions). 

So we kind of like the Enzo approach. Tim's approach still has some merit though.

\end{frame}



\begin{frame}
\frametitle{Initial Conditions}

Both the Tim and Enzo approaches here require caring about the initial conditions. 

Enzo needs to make sure that the initial conditions (startup costs etc) have attenuated before the measurement point. 

Tim needs to ensure that the initial conditions impact a sufficiently small portion of all his measurements.

\end{frame}



\begin{frame}
\frametitle{Everyone's a Winner}

If we have a nicely behaved system, the time average and the ensemble average are the same (so both Tim and Enzo can be correct). 

What is a nicely behaved system? The word for this is \alert{ergodic}. 

That probably did not help, so what is an ergodic system? 

It is a system that is positive recurrent, aperiodic, and irreducible.

\end{frame}



\begin{frame}
\frametitle{Irreducibility}

\alert{Irreducibility} means a process should be able to get from one state to any other state (where state is the number of jobs in the system). 

This means the initial state of the system does not matter. 

So if we started at 0 jobs or 10 we could still get to any state in the system (jobs at 2 or 27)...

\end{frame}



\begin{frame}
\frametitle{Positive Recurrence}

\alert{Positive recurrence} means that given an irreducible system, any state $i$ is revisited infinitely often, and the time between visits to that state are finite. 

So we can define a certain state as being a ``restart''. 

The logical choice in the case of a queue or similar is the idea of the queue being empty. 

Every time the queue gets down to zero jobs, it's a ``restart'' of sorts. 

\end{frame}



\begin{frame}
\frametitle{Positive Recurrence}

This is what makes Tim's view and Enzo's view potentially the same. 

A single long run (Tim's view) is just like a number of independent runs (Enzo's view). 

Every time we get down to zero jobs in the queue, it's a restart. 

\end{frame}



\begin{frame}
\frametitle{Aperiodicity}

The \alert{aperiodicity} condition is required for the ensemble average to make sense or exist. 

That is to say, the state of the system should not be related to the the time.

Otherwise the way Enzo chooses to sample, i.e., $t = 1000$, is potentially going to skew the result.

\end{frame}



\begin{frame}
\frametitle{Tim as Enzo}

\begin{center}
	\includegraphics[width=0.8\textwidth]{images/systemrestart.png}
\end{center}

So both Tim and Enzo are correct, given an ergodic system.


\end{frame}



\begin{frame}
\frametitle{How Long Jobs Are in the System}

We could compute either the time or ensemble average. 

\begin{center}
	Time Average = $\lim_{t\to\infty}\dfrac{\sum_{i=1}^{A(t)} T_{i}}{A(t)}$
\end{center}

where $A(t)$ is the number of arrivals by time $t$ and $T_{i}$ is the time in the system of arrival $i$. The average is taken over one sample path.


\end{frame}

\begin{frame}
\frametitle{How Long Jobs Are in the System}

\begin{center}
	Ensemble Average = $\lim_{t\to\infty}E[T_{i}]$
\end{center}

where $E[T_{i}]$ is the average time in the system of job $i$, where the average is taken over all sample paths.


\end{frame}

\end{document}

