\documentclass[letterpaper,10pt]{article}

\usepackage{titling}
\usepackage{listings}
\usepackage{url}
\usepackage{setspace}
\usepackage{subfig}
\usepackage{sectsty}
\usepackage{pdfpages}
\usepackage{colortbl}
\usepackage{multirow}
\usepackage{multicol}
\usepackage{relsize}
\usepackage{amsmath}
\usepackage{fancyvrb}
\usepackage{amsmath,amssymb,amsthm,graphicx,xspace}
\usepackage[titlenotnumbered,noend,noline]{algorithm2e}
\usepackage[compact]{titlesec}
\usepackage{XCharter}
\usepackage[T1]{fontenc}
\usepackage{enumitem}
\usepackage{tikz}
\usetikzlibrary{arrows,automata,shapes,trees,matrix,chains,scopes,positioning,calc}
\tikzstyle{block} = [rectangle, draw, fill=blue!20, 
    text width=2.5em, text centered, rounded corners, minimum height=2em]
\tikzstyle{bw} = [rectangle, draw, fill=blue!20, 
    text width=4em, text centered, rounded corners, minimum height=2em]

\newcommand{\CPP}{C\nolinebreak\hspace{-.05em}\raisebox{.4ex}{\tiny\bf +}\nolinebreak\hspace{-.10em}\raisebox{.4ex}{\tiny\bf +}}
\def\CPP{{C\nolinebreak[4]\hspace{-.05em}\raisebox{.4ex}{\tiny\bf ++}}}

\let\LaTeXtitle\title
\renewcommand{\title}[1]{\LaTeXtitle{\textsf{#1}}}


\addtolength{\oddsidemargin}{-1.000in}
\addtolength{\evensidemargin}{-0.500in}
\addtolength{\textwidth}{2.0in}
\addtolength{\topmargin}{-1.000in}
\addtolength{\textheight}{1.75in}
\addtolength{\parskip}{\baselineskip}
\setlength{\parindent}{0in}
\renewcommand{\baselinestretch}{1.5}

\singlespace


\begin{document}

\lecture{18 --- Inlining, HLL}{\term}{Patrick Lam}

\section*{Good Programming Practices: Inlining}
We have seen the notion of inlining:
  \begin{itemize}
    \item Instructs the compiler to just insert the function code in-place,
      instead of calling the function.
    \item Hence, no function call overhead!
    \item Compilers can also do better---context-sensitive---operations they couldn't
      have done before.
  \end{itemize}

OK, so inlining removes overhead. Sounds like better performance! Let's inline everything!
There are two ways of inlining in C++.

\paragraph{Implicit inlining.} (defining a function inside a class definition):
  \begin{lstlisting}[language=C]
class P {
public:
    int get_x() const { return x; }
...
private:
    int x;
};
  \end{lstlisting}

\paragraph{Explicit inlining.} Or, we can be explicit:
  \begin{lstlisting}[language=C]
inline max(const int& x, const int& y) {
    return x < y ? y : x;
}
  \end{lstlisting}

\paragraph{The Other Side of Inlining.}
Inlining has one big downside:
  \begin{itemize}
    \item Your program size is going to increase.
  \end{itemize}
   This is worse than you think:
      \begin{itemize}
        \item Fewer cache hits.
        \item More trips to memory.
      \end{itemize}
   Some inlines can grow very rapidly (C++ extended constructors).
  Just from this your performance may go down easily.

  Note also that inlining is merely a suggestion to compilers~\cite{gcc:inlining}.
  They may ignore you.
  For example:
  \begin{itemize}
    \item taking the address of an ``inline'' function and using it; or
    \item virtual functions (in C++),
  \end{itemize}
  will get you ignored quite fast.

\paragraph{Implications of inlining.} Inlining can make your life worse in two ways.
First, debugging is more difficult (e.g. you can't set a breakpoint in a function that
  doesn't actually exist).
 Most compilers simply won't inline code with debugging symbols on.
 Some do, but typically it's more of a pain.

Second, it can be a problem for library design:
  \begin{itemize}
    \item If you change any inline function in your library, any users
      of that library have to {\bf recompile} their program if the
      library updates. (Congratulations, you made a non-binary-compatible change!)
  \end{itemize}
This would not be a problem for non-inlined functions---programs execute the new function
dynamically at runtime.

\section*{High-Level Language Performance Tweaks}
So far, we've only seen C---we haven't seen anything complex, and C is
low level, which is good for learning what's really going on.

 Writing compact, readable code in C is hard, especially when \#define
macros and {\tt void *} beckon.

    C++11 has made major strides towards readability and
    efficiency---it provides light-weight abstractions. We'll look at
    a couple of examples.

\paragraph{Sorting.} Our goal is simple: we'd like to sort a bunch of integers.
In C, you would usually just use qsort from {\tt stdlib.h}.

  \begin{lstlisting}
void qsort (void* base, size_t num, size_t size,
            int (*comparator) (const void*, const void*));
  \end{lstlisting}

This is a fairly ugly definition (as usual, for generic C functions). How ugly is it?
Let's look at a usage example.
  \begin{lstlisting}[language=C]
#include <stdlib.h>

int compare(const void* a, const void* b)
{
    return (*((int*)a) - *((int*)b));
}

int main(int argc, char* argv[])
{
    int array[] = {4, 3, 5, 2, 1};
    qsort(array, 5, sizeof(int), compare);
}
  \end{lstlisting}
This looks like a nightmare, and is more likely to have bugs than what we'll see next.


C++ has a sort with a much nicer interface\footnote{\ldots well, nicer to use, after you get over templates.}:

  \begin{lstlisting}[language=C++]
template <class RandomAccessIterator>
void sort (
    RandomAccessIterator first,
    RandomAccessIterator last
);

template <class RandomAccessIterator, class Compare>
void sort (
    RandomAccessIterator first,
    RandomAccessIterator last,
    Compare comp
);
  \end{lstlisting}
It is, in fact, easier to use:
  \begin{lstlisting}[language=C++]
#include <vector>
#include <algorithm>

int main(int argc, char* argv[])
{
    std::vector<int> v = {4, 3, 5, 2, 1};
    std::sort(v.begin(), v.end());
}
  \end{lstlisting}

{\bf Note:} Your compare function can be a function or a functor. (Don't know what functors
are? In C++, they're functions with state.) By default,
  {\tt sort} uses {\tt operator$<$} on the objects being sorted.

  \begin{itemize}
    \item Which is less error prone?
    \item Which is {\bf faster}?
  \end{itemize}

The second question is empirical. Let's see. We generate an array of 2 million ints
and sort it (10 times, taking the average).

\begin{itemize}
\item qsort: 0.49 seconds
\item C++ sort: 0.21 seconds
\end{itemize}

The C++ version is {\bf twice} as fast. Why?
      \begin{itemize}
        \item The C version just operates on memory---it has no clue about the
          data.
        \item We're throwing away useful information about what's being sorted.
        \item A C function-pointer call prevents inlining of the compare function.
      \end{itemize}
OK. What if we write our own sort in C, specialized for the data?

\begin{itemize}
\item Custom C sort: 0.29 seconds
\end{itemize}

Now the C++ version is still faster (but it's close). But, this is
quickly going to become a maintainability nightmare.
      \begin{itemize}
        \item Would you rather read a custom sort or 1 line?
        \item What (who) do you trust more?
      \end{itemize}

\subsection*{Lesson}
Abstractions will not make your program slower. 

\noindent
They allow speedups and are much easier to maintain and read.

\subsection*{Vectors vs Lists}
Consider two
problems.

\begin{enumerate}
\item Generate {\bf N} random integers and insert them into (sorted)
      sequence.
      
      {\bf Example:} 3 4 2 1
      
      \begin{itemize}
        \item 3
        \item 3 4
        \item 2 3 4
        \item 1 2 3 4
      \end{itemize}

\item Remove {\bf N} elements one-at-a-time by going to a random position
      and removing the element.

      {\bf Example:} 2 0 1 0
      
      \begin{itemize}
        \item 1 2 4
        \item 2 4
        \item 2
        \item 
      \end{itemize}
\end{enumerate}

For which {\bf N} is it better to use a list than a vector (or array)?

 
\paragraph{Complexity analysis.} As good computer scientists, let's analyze
the complexity.  

{\bf Vector}:\\[-2em]
      \begin{itemize}
        \item Inserting\\[-2em]
          \begin{itemize}
            \item $O(\log n)$ for binary search
            \item $O(n)$ for insertion (on average, move half the elements)
          \end{itemize}
        \item Removing\\[-2em]
          \begin{itemize}
            \item $O(1)$ for accessing
            \item $O(n)$ for deletion (on average, move half the elements)
          \end{itemize}
      \end{itemize}

{\bf List}:\\[-2em]
      \begin{itemize}
        \item Inserting\\[-2em]
          \begin{itemize}
            \item $O(n)$ for linear search
            \item $O(1)$ for insertion
          \end{itemize}
        \item Removing\\[-2em]
          \begin{itemize}
            \item $O(n)$ for accessing
            \item $O(1)$ for deletion
          \end{itemize}
      \end{itemize}

Therefore, based on their complexity, lists should be better.

\paragraph{Reality.} OK, here's what happens. 
\begin{verbatim}
$ ./vector_vs_list 50000
Test 1
======
vector: insert 0.1s   remove 0.1s   total 0.2s
list:   insert 19.44s   remove 5.93s   total 25.37s
Test 2
======
vector: insert 0.11s   remove 0.11s   total 0.22s
list:   insert 19.7s   remove 5.93s   total 25.63s
Test 3
======
vector: insert 0.11s   remove 0.1s   total 0.21s
list:   insert 19.59s   remove 5.9s   total 25.49s
\end{verbatim}

{\bf Vectors} dominate lists, performance wise. Why?
  \begin{itemize}
    \item Binary search vs. linear search complexity dominates.
    \item Lists use far more memory.
      {\bf On 64 bit machines:}
      \begin{itemize}
        \item Vector: 4 bytes per element.
        \item List: At least 20 bytes per element.
      \end{itemize}
    \item Memory access is slow, and results arrive in blocks:
      \begin{itemize}
        \item Lists' elements are all over memory, hence many
          cache misses.
        \item A cache miss for a vector will bring a lot more usable data.
      \end{itemize}
  \end{itemize}

So, here are some tips for getting better performance.
  \begin{itemize}
    \item Don't store unnecessary data in your program.
    \item Keep your data as compact as possible.
    \item Access memory in a predictable manner.
    \item Use vectors instead of lists by default.
    \item Programming abstractly can save a lot of time.
    \item Often, telling the compiler more gives you better code.
    \item Data structures can be critical, sometimes more than complexity.
    \item {\bf Low-level code != Efficient}.
    \item Think at a low level if you need to optimize anything.
    \item Readable code is good code---different hardware needs different
      optimizations.
  \end{itemize}


\bibliographystyle{alpha}
\bibliography{459}


\end{document}
