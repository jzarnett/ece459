\documentclass[letterpaper,10pt]{article}

\usepackage{titling}
\usepackage{listings}
\usepackage{url}
\usepackage{setspace}
\usepackage{subfig}
\usepackage{sectsty}
\usepackage{pdfpages}
\usepackage{colortbl}
\usepackage{multirow}
\usepackage{multicol}
\usepackage{relsize}
\usepackage{amsmath}
\usepackage{fancyvrb}
\usepackage{amsmath,amssymb,amsthm,graphicx,xspace}
\usepackage[titlenotnumbered,noend,noline]{algorithm2e}
\usepackage[compact]{titlesec}
\usepackage{XCharter}
\usepackage[T1]{fontenc}
\usepackage{enumitem}
\usepackage{tikz}
\usetikzlibrary{arrows,automata,shapes,trees,matrix,chains,scopes,positioning,calc}
\tikzstyle{block} = [rectangle, draw, fill=blue!20, 
    text width=2.5em, text centered, rounded corners, minimum height=2em]
\tikzstyle{bw} = [rectangle, draw, fill=blue!20, 
    text width=4em, text centered, rounded corners, minimum height=2em]

\newcommand{\CPP}{C\nolinebreak\hspace{-.05em}\raisebox{.4ex}{\tiny\bf +}\nolinebreak\hspace{-.10em}\raisebox{.4ex}{\tiny\bf +}}
\def\CPP{{C\nolinebreak[4]\hspace{-.05em}\raisebox{.4ex}{\tiny\bf ++}}}

\let\LaTeXtitle\title
\renewcommand{\title}[1]{\LaTeXtitle{\textsf{#1}}}


\addtolength{\oddsidemargin}{-1.000in}
\addtolength{\evensidemargin}{-0.500in}
\addtolength{\textwidth}{2.0in}
\addtolength{\topmargin}{-1.000in}
\addtolength{\textheight}{1.75in}
\addtolength{\parskip}{\baselineskip}
\setlength{\parindent}{0in}
\renewcommand{\baselinestretch}{1.5}

\singlespace


\begin{document}

\lecture{33 --- Operational Laws \& M/M/k Servers }{\term}{Jeff Zarnett}

The purpose of the examination in the recent lecture of probability is so that we could able to answer some interesting ``what if'' questions. So let's take a look at some of those, but first we'll stop to discuss a few operational laws and principles.

\section*{Little's Law and Other Operational Laws}
Little's Law is a very famous result, saying that the average number of jobs in the system equals the product the average arrival rate into the system and the average arrival time in the system. The source on this section is~\cite{pmd}. 

\paragraph{Open Systems.} Let's start with an open system. The law, written more formally:

\begin{center}
	$E[N] = \lambda E[T]$
\end{center}

Where $E[N]$ is the expected value of the number of jobs in the system. $\lambda$ is the average arrival rate into the system. $E[T]$ is the mean time jobs spend in the system. The basic setup of Little's Law looks something like this~\cite{pmd}:

\begin{center}
	\includegraphics[width=0.5\textwidth]{images/littleslaw.png}
\end{center}

Note that we don't need to know anything about the arrival process (Bernoulli, Poisson, etc...), the service time distribution, network topology, etc. It seems intuitive that this is the case (or it should). Imagine a fast food restaurant: they make money by quick turnaround time, so they get people out of the place quickly (low $E[T]$) and accordingly they don't require a lot of seating (low $E[N]$). A sit down restaurant is the opposite though; people leave slowly (high $E[T]$) and therefore the restaurant needs lots of seating (more $E[N]$). This example might seem weird from the perspective of the customer though -- from your perspective, you may want to enjoy your evening; but the restaurant is eager to turn your table over: get you out of there so a new set of guests can be seated.

If you prefer to think of this in a single FCFS queue version, imagine a customer arrives and sees $E[N]$ jobs ahead of her in the queue. The expected time for each customer to complete is $1/\lambda$, because the average rate of completions is $\lambda $. So we can approximate $E[T]$ as being roughly $\dfrac{1}{\lambda}E[N]$.

\paragraph{Closed Systems.} Remember that for closed systems, we have a rule that says there is $N$ jobs in process at any given time (the multiprocessing level of the system). If the system is ergodic, then $N = X \cdot E[T]$ where $N$ is the multiprogramming level, $X$ is the throughput rate, and $E[T]$ is the mean time jobs spend in the system. This assumes that there is zero think time, i.e., that jobs are always ready at once and don't have to wait for silly users.

If we do have to deal with the vagaries of users and this time, then we care more about the response time $E[R]$. So for a terminal-driven system, the expected response time is $E[R] = \dfrac{N}{X} - E[Z]$ where $N$ is the multiprogramming level, $X$ is the throughput, and $E[Z]$ is the mean time spent thinking. 

\section*{M/M/1}

Probabilistic processes are described according to their models, which will probably one of the three~\cite{swps}:

\begin{enumerate}
	\item Deterministic (D) -- The process is predictable and characterized by whatever constant factors. For example, the inter arrival times are constant (e.g., a task arrives every minute.)
	\item Markov (M) -- A memoryless process; the future states of the process are independent of the past history. The future state depends on only the present state.
	\item General (G) -- Completely arbitrary.
\end{enumerate}

Right, we're going to focus on Markov processes, because they are nicer (and we have only limited time). It means that the number of arrivals follow the Poisson distribution; the inter arrival times follow the exponential distribution, and service times follow exponential distribution too. 

Those letters we saw are part of Kendall notation. It has six symbols, written in a specific order, separated by slashes. The order is $\alpha / \sigma / m / \beta / N / Q$. See the table below for the full explanation:

\begin{center}
\begin{tabular}{l|l} 
	\textbf{Symbol} & \textbf{Meaning} \\ \hline
	$\alpha$ & The type of distribution (Markov, General, Deterministic) \\ \hline
	$\sigma$ & The type of probability distribution for service time \\ \hline
	$m$ & Number of servers \\ \hline
	$\beta$ & Buffer size \\ \hline
	$N$ & Allowed population size (finite or infinite) \\ \hline
	$Q$ & Queueing policy \\ 
\end{tabular}
\end{center}

We often leave off the last three, assuming that there is an infinite buffer, infinite population, and a FIFO queueing policy. If that is the case, then we have only three values. Those three then produce the ``M/M/1'' and ``M/M/k'' symbols. ``M/M/1'' means a Markov arrival process, exponential queueing system, and a single server. When there are $k$ servers, then, of course the 1 is replaced with the $k$. These are the systems that we are going to examine.

We should also think about utilization, denoted $\rho$ in the notation. It is a fraction between 0 and 1 and it is simply the amount of time that the server is busy. We talked about this earlier in an informal way, but now we can actually calculate it: $\rho = \lambda \times s$ (the arrival rate and service time). 

For M/M/1 systems, the completion time average $T_{q}$ is $\dfrac{s}{(1-\rho)}$ and the average length of the queue $W$ is $\dfrac{\rho^{2}}{1-\rho}$.

An example from~\cite{williams-q}: we have a server that completes a request, on average, in 10~ms. The time to complete a request is exponentially distributed. Over a period of 30 minutes, 117~000 jobs arrive. So this is a M/M/1 situation. How long did it take to complete the average request? What is the average length of the queue?

The service time $s$ is given as $0.01s$, the arrival rate is 65 requests per second. So we can calculate $\rho = 0.01 \times 65 = 0.65$. So we have what we need to plug and chug using the formulae from above to find the time to complete the average request is 28.6~ms and the average length of the queue is 1.21.

What about the number of jobs in the system? The value $Q$ gives the average number of jobs, including the waiting jobs and the ones being served. It is an average, of course. The probability that there are exactly $x$ jobs in the system at any time is given by the formula: $(1-\rho)\rho^{x}$. The probability that the number of jobs is less than or equal to $n$ is then given by: $\sum\limits_{i=0}^{n}(1-\rho)\rho^{i}$ (the sum of the probabilities of each of the numbers from 0 up to $n$). If you want to know the probability that there are more then $n$ at a time, then you can compute the sum from $n-1$ up to infinity. That might be unpleasant to calculate, but remember that probabilities sum to 1, so you can say that the probability of more than $n$ requests at once is simply $1 - \sum\limits_{i=0}^{n}(1-\rho)\rho^{i}$.

\section*{M/M/k}

Now let us take it to multiple servers. We will say jobs arrive at a single queue and then when a server is ready it will take the first job from the front of the queue. The servers are identical and jobs can be served by any server. So far, so simple.

Sadly, the math just got harder. Let's turn again to~\cite{williams-q} as the source for this section. The server utilization for the server farm is now $\rho = \lambda s / N$; the average utilization for all $N$ servers. To make our calculations a little easier, we want an intermediate value $K$ which looks scary, but is not so bad:

\begin{center}
	$K = \cfrac{\sum_{i=0}^{N-1}\dfrac{(\lambda s)^{i}}{i!}}{{\sum_{i=0}^{N}\dfrac{(\lambda s)^{i}}{i!}}}$
\end{center}

The first term, $i = 0$, is always 1. The denominator is always larger than the numerator, so $K$ is always less than 1. $K$ has no intrinsic meaning, it is just a a computational shorthand so the other formulae are not so messy.

What is the probability that all servers are busy? We represent this as $C$, the probability a new job will have to wait in the queue.

\begin{center}
	$C = \cfrac{1 - K}{1 - \dfrac{\lambda s K}{N}}$
\end{center}

The M/M/k formulae, then, for the average completion time and average length of the queue are:

\begin{center}
  $T_{q} = \dfrac{C s}{k(1 - \rho)} + s$ \qquad\qquad\qquad\qquad $W = C \dfrac{\rho}{1 - \rho}$
\end{center}

Let's do an example. Suppose we have a printer that can complete an average print job in two minutes. Every 2.5 minutes, a user submits a job to the printer. How long does it take to get the print job on average? We're starting with a single printer, so the system is M/M/1. Service time $s$ is 2 minutes; the arrival rate $\lambda$ is $1/2.5 = 0.4$. So $\rho = \lambda \times s = 0.4 \times 2 = 0.8$. So $T_{q} = s / (1 - \rho ) = 2 / (1 -0.8 ) = 10$. So ten minutes to get the print job. Ouch.

Here we have an opportunity to use the predictive power of queueing theory. Management is convinced that ten minute waits for print jobs is unreasonable, so we have been asked to decide what to do: should we buy a second printer of the same speed, or should we sell the old one and buy a printer that is double the speed?

The faster printer calculation is easy enough. Now $s = 1.0$ and $\lambda$ remains $0.4$, making $\rho = 0.4$. So rerunning the calculation: $T_{q} = s / (1 - \rho ) = 1 / (1 - 0.4 ) = 1.67$. 1:40 is a lot less time than 10:00! 

The two printer solution is more complicated. So let us calculate $K$ as the intermediate value. 

\begin{center}
	$K = \cfrac{\sum_{i=0}^{N-1}\dfrac{(\lambda s)^{i}}{i!}}{{\sum_{i=0}^{N}\dfrac{(\lambda s)^{i}}{i!}}} = \cfrac{\dfrac{(\lambda s)^{0}}{0!} + \dfrac{(\lambda s)^{1}}{1!}}{\dfrac{(\lambda s)^{0}}{0!} + \dfrac{(\lambda s)^{1}}{1!} + \dfrac{(\lambda s)^{2}}{2!}} = 0.849057$
	
\end{center}

Now we can calculate $C$ as 0.22857 and $T_{q}$ as 2.57 minutes (by simple plug and chug calculations given the formulae above). Two observations jump out at us: (1) we doubled the number of printers, but now jobs are completed almost four times faster; and (2) the single fast printer is better, if utilization is low.

That is an important condition: if utilization is low. At some point will the two printers be a better choice than the single fast one? What if both printers are used to the max (100\% load)...?

\section*{Making Predictions}

\cite{williams-perf}.

\bibliographystyle{alpha}
\bibliography{459}


\end{document}
