\documentclass[letterpaper,10pt]{article}

\usepackage{titling}
\usepackage{listings}
\usepackage{url}
\usepackage{setspace}
\usepackage{subfig}
\usepackage{sectsty}
\usepackage{pdfpages}
\usepackage{colortbl}
\usepackage{multirow}
\usepackage{multicol}
\usepackage{relsize}
\usepackage{amsmath}
\usepackage{fancyvrb}
\usepackage{amsmath,amssymb,amsthm,graphicx,xspace}
\usepackage[titlenotnumbered,noend,noline]{algorithm2e}
\usepackage[compact]{titlesec}
\usepackage{XCharter}
\usepackage[T1]{fontenc}
\usepackage{enumitem}
\usepackage{tikz}
\usetikzlibrary{arrows,automata,shapes,trees,matrix,chains,scopes,positioning,calc}
\tikzstyle{block} = [rectangle, draw, fill=blue!20, 
    text width=2.5em, text centered, rounded corners, minimum height=2em]
\tikzstyle{bw} = [rectangle, draw, fill=blue!20, 
    text width=4em, text centered, rounded corners, minimum height=2em]

\newcommand{\CPP}{C\nolinebreak\hspace{-.05em}\raisebox{.4ex}{\tiny\bf +}\nolinebreak\hspace{-.10em}\raisebox{.4ex}{\tiny\bf +}}
\def\CPP{{C\nolinebreak[4]\hspace{-.05em}\raisebox{.4ex}{\tiny\bf ++}}}

\let\LaTeXtitle\title
\renewcommand{\title}[1]{\LaTeXtitle{\textsf{#1}}}


\addtolength{\oddsidemargin}{-1.000in}
\addtolength{\evensidemargin}{-0.500in}
\addtolength{\textwidth}{2.0in}
\addtolength{\topmargin}{-1.000in}
\addtolength{\textheight}{1.75in}
\addtolength{\parskip}{\baselineskip}
\setlength{\parindent}{0in}
\renewcommand{\baselinestretch}{1.5}

\singlespace


\begin{document}

\lecture{ 34 --- DevOps for P4P }{\term}{Patrick Lam \& Jeff Zarnett}

Two topics today: 1) DevOps considerations (think
big); 2) the cost of scalability (think small).

\section*{DevOps for P4P}
% this part took the whole 50 minutes, at least on the blackboard.
So far, we've talked almost exclusively about one-off computations:
you want to figure out the answer to a question, and you write code to
do that. Our assignments have been like that, for instance. But a lot
of the time we want to keep systems running over time. That gets us
into the notion of operations. Your service or product is more likely than 
ever to have at least some component that's server side (whether hosted
in a cloud service or not) that's under your control.

The theme today will be using software development skills in
operations (e.g., system administration, database management, etc). This
does have some relevance, because the operations (or IT, if you prefer)
processes and procedures, while different from development, have some
similarities. 

Even when we've talked about multi-computer tools like MPI and cloud
computing, it still has not been in the context of keeping your
systems operational over longer timescales. The trend today is away
from strict separation between a development team, which writes the
software, and an operations team, which runs the software.

\begin{center}
	\includegraphics[width=0.45\textwidth]{images/devops.jpg}
\end{center}

The separation is totally nonexistent at the typical startup company. There isn't the money to pay for separate developers and operations teams. And in the beginning there's probably not that many servers, just a few demo systems, test systems, etc... but it spirals out from there. You're not really going to ask the sales guys to manage these servers, are you? So, there's DevOps. 

Is DevOps a good idea? Like most ideas it can be used for both good and evil. There's a lot to be said for letting the developers be involved in all the parts of the software from development to deployment to management to training the customers. Developers can learn a lot by having to do these kinds of things, and be motivated to make proper management and maintenance tools and procedures. If we make the pain of operations felt by developers, they might do something about it. If it's the problem of another team, somehow those tickets just never make it to the top of the backlog.

Thanks to Chris Jones and Niall Murphy for the following points.

\subsection*{Configuration as code}
Systems have long come with complicated configuration options.
Sendmail is particularly notorious, but apache and nginx aren't super
easy to configure either.\footnote{If anyone should be foolish enough to want to look into procmail, well, good luck to you...} The first principle is to treat \emph{configuration as code}.
Therefore:
\begin{itemize}
\item use version control on your configuration.
\item implement code reviews on changes to the configuration.
\item test your configurations: that means that you check that they
  generate expected files, or that they spawn expected
  services. (Behaviours, or outcomes.) Also, configurations should
  ``converge''. Unlike code, they might not terminate; we're talking
  indefinitely-running services, after all. But the CPU usage should
  go down after a while, for instance.
\item aim for a suite of modular services that integrate together smoothly.
\item refactor configuration files (Puppet manifests, Chef recipes, etc);
\item use continuous builds (more on that later).
\end{itemize}

Furthermore, it's an excellent idea to have tools for configuration. It's not enough to just have a wiki page or github document titled ``How to Install AwesomeApp'' (fill in name of program here). There are tons of great tools like the Red Hat Package Manager (RPM) which will allow you to make the installation and update process automatic and simple. Complicated means mistakes... people forget steps. They are human. Don't let them make mistakes: make it automatic.

What about containers? Well, let's see how we got there first. In the beginning, you had services where you installed the binaries and config files by hand. That sucked. So there were packages; a package includes everything the program needs (including a list of dependencies) and a script to install it and set it up. Great! But if you just install multiple services on the same machine you don't get isolation and you might have incompatible versions of dependencies and you're in RPM hell (see also: JAR hell, classloader hell, and DLL hell).

Right, so instead you say you should have virtual machines: you configure the VM parameters and install the guest OS and set it up (and you can copy-paste the initial image, which helps) but for every application you have a guest operating system running underneath. Maybe we don't need every app to have its own guest OS; why do we have to install the same security patch ten times...?

Containerization gives many of the advantages of this separation, but without nearly so much overhead of the guest operating systems (both its maintenance and runtime costs). Containers are run by a container engine so there is some abstraction of the underlying hardware, and the container is assembled from a specification that says what libraries, tools, etc. are needed. And thus when the container is built and deployed it is sufficiently isolated but shares (in read only mode) where it can. So a container is a very lightweight VM, in some sense. See this diagram from~\cite{netappcontainer}:

\begin{center}
	\includegraphics[width=0.55\textwidth]{images/cvm.png}
\end{center}


\subsection*{Servers as cattle, not pets}
By servers, I mean servers, or virtual machines, or containers. It's much better to have a reproducible process for deployment of a server than doing it manually every single time. The amount of manual intervention should be minimized and ideally zero. If this is done you can save a lot of hours of time, reduce errors, and allow for automatic scaling (starting and stopping servers depending on demand).

The title references the idea that cattle are dealt with as a herd: you try to get the whole group to move along and do what they need. Pets are individuals, though, and you'll treat them all differently. This amount of individual attention quickly becomes unmanageable and there's no reason why you should worry about these differences in a world with virtualization (containers) or similar. 

\subsection*{Common infrastructure}
Use APIs to access your infrastructure. That is, you should view different parts of your infrastructure as having an interface and communication is done exclusively via the interface/API. This reduces the coupling between different components, and, as we've discussed, allows you to scale the parts that need scaling. 

Try to avoid not-invented-here syndrome: it is usually better to use an open-source tool (or a commercial one) than to roll your own. Some examples might be:
\begin{itemize}
\item storage: some sort of access layer (e.g., MongoDB or S3);
\item naming and discovery infrastructure (e.g., Consul) (more below);
\item monitoring infrastructure (e.g., Prometheus).
\end{itemize}

However, be prepared to build your own tools if needed. Sometimes what you want, or need, doesn't exist (yet). Think carefully about whether this service that is needed is really part of your core competence and whether creating it adds sufficient value to the business. It's fun to make your own system and all, but are you doing what you're best at? 

Think extra carefully if you plan to do roll your own anything that is security or encryption related. I'm just going to say that unless you have experts on staff who know the subject really well and you're willing to pay for external audits and the like, you're more likely to end up with a terrible security breach than a terrific secure system. PS: a breach of data protection regulations can get very expensive. See \url{https://www.enforcementtracker.com/} to find out which companies have recently gotten their wrists slapped.

As a second followup soapbox point to that: if what you are looking for doesn't exist, there might be a reason, Maybe the reason is that you are the first to think of it, but consider the possibility that it's not that good of an idea (either due to inefficiency or just not being great in principle).  

\subsection*{Design for 10$\times$ growth, redesign before 100$\times$}
[original credit: Jeff Dean at Google] This discussion is based on
Martin Fowler's piece on sacrificial architecture:
\url{http://martinfowler.com/bliki/SacrificialArchitecture.html}.

Consider eBay: in 1995, perl scripts; in 1997, C++/Windows; in 2002,
Java.  Each of these architectures was appropriate at the time, but
not as the requirements change. The more sophisticated successor
architectures, however, would have been overkill at an earlier
time. And it's hard to predict what would be needed in the future.
% look up eBay growth stats
%http://martinfowler.com/bliki/images/sacrificialArchitecture/sketch.png
%http://web.archive.org/web/20000510004517/http://www.ebay.com/
% http://www.ebay.com/

\begin{quote}
``Perf is a feature''.\\
\hfill --- Jeff Atwood
\end{quote}
That is, you apply developer time to perf, and you make engineering tradeoffs
to get it. Some thoughts:
\begin{itemize}
\item design with the eventual replacement in mind;
\item don't abandon internal quality (e.g. modularity);
\item sacrifice individual modules at a time, not the whole system;
\item you can also implement new features with a rough draft and deploy to a test audience.
\end{itemize}

\subsection*{Naming}
Naming is one of the hard problems in computing. There is a saying that there are only two hard things in computers: cache invalidation, naming things, and off by one errors. 

There are a
lot of ways to name things. We'll talk about
systems/VMs\footnote{\url{http://mnx.io/blog/a-proper-naming-scheme}},
but naming is necessary for resources of all kinds.

In brief:
\begin{itemize}
\item use canonical one-word names for servers;
\item but, use aliases to specify functions, e.g. 1) geography (nyc); 2) environment (dev/tst/stg/prod); 
3) purpose (app/sql/etc); and 4) serial number.
\end{itemize}
This enables you to have a way of referring to each machine in an absolute sense, but also 
allows you to use functional names when creating dependencies between systems.

There's also the Java package approach of infinite dots: live.application.customer.webdomain.com or however you want to call it. But pick something and be consistent.

\subsection*{Other Topics}
Beyond the five principles above, there are a couple more techniques that particularly apply to
DevOps:

\paragraph{Continuous Integration.} 
This is now a best practice. It's enabled by the use of version control, good tests, and scripted deployments.
It works like this:
\begin{itemize}
\item pull code from version control;
\item build;
\item run tests;
\item report results.
\end{itemize}
What's also key is a social convention to not break the build. These things get done automatically on every commit and the results are sent to people by e-mail or instant messenger (because e-mail is for old people, right?).\footnote{I did work at a company where the person who broke the build got a sign outside his cubicle that said IOTD - Idiot of the Day. I'm not too proud to admit that I won this award on my last day of the co-op term.}

%https://jenkins-ci.org/sites/default/files/jenkins_logo.png

CI is good for all code, but it's especially good for configuration-as-code, which is especially likely
to break in different environments.

\paragraph{Canarying.}
%http://www.post-gazette.com/image/2013/10/29/ca27,76,1566,1822/Canary.jpg
Deploy new software incrementally alongside production software, also known as ``test in prod''. Sometimes
you just don't know how code is really going to work until you try it. After, of course, you use your best
efforts to make sure the code is good. Steps:
\begin{itemize}
\item stage for deployment;
\item remove canary servers from service;
\item upgrade canary servers;
\item run automatic tests on upgraded canaries;
\item reintroduce canary servers into service;
\item see how it goes!
\end{itemize}
Of course, you should implement your system so that rollback is possible.

\paragraph{Monitoring.}

Monitoring is surprisingly difficult. There are a lot of recommendations about what to monitor and what to do about it. We care about performance so here are a few things to think about:

\begin{itemize}
	\item CPU Load
	\item Memory Utilization
	\item Disk Space
	\item Disk I/O
	\item Network Traffic
	\item Clock Skew
	\item Application Response Times
\end{itemize}

With multiple systems, you will want some sort of dashboard that gives an overview of all the multiple systems in a summary. The summary needs to be sufficiently detailed that you can detect if anything is wrong, but not an overwhelming wall of data. Then you do not necessarily want to pay someone to stare at the dashboard and press the  ``Red Alert!'' button if anything goes out of some preset range of what is okay. No, for that we need some automatic monitoring.

Here's one way to think about it. 
% track down the source here: Niall Murphy interview.
\begin{itemize}
\item {\bf Alerts}: a human must take action now;
\item {\bf Tickets}: a human must take action soon (hours or days);
\item {\bf Logging}: no need to look at this except for forensic/diagnostic purposes.
\end{itemize}
A common bad situation is logs-as-tickets: you should never be in the
situation where you routinely have to look through logs to find
errors. Write code to scan logs.

It is very important to be judicious about the use of alerts. If your alerts are too common, they get ignored. When you hear the fire alarm in a building, chances are your thought is not ``the building is on fire; I should leave it immediately in an orderly fashion.''. More likely your reaction is ``great, some jerk\footnote{This is the PG-13 version of what I actually think.} has pulled the fire alarm for a stupid prank or to get out of failing a midterm.'' This is because we have been trained by far too many false alarms to think that any alarm is a false one. It's a good heuristic; you'll be correct most of the time. But if there is an actual fire, you will not only be wrong, you might also be dead.

Still, alerts and tickets are a great way to make user pain into developer pain. Being woken up in the middle of the night (... day? A lot of programmers are nocturnal, now that I think of it) because of some SUPER CRITICAL ticket OMG KITTENS ARE ENDANGERED is an excellent way to learn the lesson that production code needs to be written carefully, reviewed, QA'd, and perhaps run by a customer or two before it gets deployed to everyone. Developers, being human (... grant me some leeway here), will probably take steps to avoid their pain\footnote{There is a great quotation to this effect by Fr\'ed\'eric Bastiat about how men will avoid pain and work is pain.}. and they will take steps that keep these things from happening in the future: good processes and monitoring and all that goes with it.


\bibliographystyle{alpha}
\bibliography{459}


\end{document}
