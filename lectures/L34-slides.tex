
\documentclass[letterpaper,hide notes,xcolor={table,svgnames},pdftex,10pt]{beamer}
\def\showexamples{t}


%\usepackage[svgnames]{xcolor}

%% Demo talk
%\documentclass[letterpaper,notes=show]{beamer}

\usecolortheme{crane}
\setbeamertemplate{navigation symbols}{}

\usetheme{MyPittsburgh}
%\usetheme{Frankfurt}

%\usepackage{tipa}

\usepackage{hyperref}
\usepackage{graphicx,xspace}
\usepackage[normalem]{ulem}
\usepackage{multicol}

\newcommand\SF[1]{$\bigstar$\footnote{SF: #1}}

\usepackage[default]{sourcesanspro}
\usepackage[T1]{fontenc}

\newcounter{tmpnumSlide}
\newcounter{tmpnumNote}

% old question code
%\newcommand\question[1]{{$\bigstar$ \small \onlySlide{2}{#1}}}
% \newcommand\nquestion[1]{\ifdefined \presentationonly \textcircled{?} \fi \note{\par{\Large \textbf{?}} #1}}
% \newcommand\nanswer[1]{\note{\par{\Large \textbf{A}} #1}}


 \newcommand\mnote[1]{%
   \addtocounter{tmpnumSlide}{1}
   \ifdefined\showcues {~\tiny\fbox{\arabic{tmpnumSlide}}}\fi
   \note{\setlength{\parskip}{1ex}\addtocounter{tmpnumNote}{1}\textbf{\Large \arabic{tmpnumNote}:} {#1\par}}}

\newcommand\mmnote[1]{\note{\setlength{\parskip}{1ex}#1\par}}

%\newcommand\mnote[2][]{\ifdefined\handoutwithnotes {~\tiny\fbox{#1}}\fi
% \note{\setlength{\parskip}{1ex}\textbf{\Large #1:} #2\par}}

%\newcommand\mnote[2][]{{\tiny\fbox{#1}} \note{\setlength{\parskip}{1ex}\textbf{\Large #1:} #2\par}}

\newcommand\mquestion[2]{{~\color{red}\fbox{?}}\note{\setlength{\parskip}{1ex}\par{\Large \textbf{?}} #1} \note{\setlength{\parskip}{1ex}\par{\Large \textbf{A}} #2\par}\ifdefined \presentationonly \pause \fi}

\newcommand\blackboard[1]{%
\ifdefined   \showblackboard
  {#1}
  \else {\begin{center} \fbox{\colorbox{blue!30}{%
         \begin{minipage}{.95\linewidth}%
           \hspace{\stretch{1}} Some space intentionally left blank; done at the blackboard.%
         \end{minipage}}}\end{center}}%
         \fi%
}



%\newcommand\q{\tikz \node[thick,color=black,shape=circle]{?};}
%\newcommand\q{\ifdefined \presentationonly \textcircled{?} \fi}

\usepackage{listings}
\lstset{%
  keywordstyle=\bfseries,
  aboveskip=15pt,
  belowskip=15pt,
  captionpos=b,
  identifierstyle=\ttfamily,
  escapeinside={(*@}{@*)},
  stringstyle=\ttfamiliy,
  frame=lines,
  numbers=left, basicstyle=\scriptsize, numberstyle=\tiny, stepnumber=0, numbersep=2pt}

\usepackage{siunitx}
\newcommand\sius[1]{\num[group-separator = {,}]{#1}\si{\micro\second}}
\newcommand\sims[1]{\num[group-separator = {,}]{#1}\si{\milli\second}}
\newcommand\sins[1]{\num[group-separator = {,}]{#1}\si{\nano\second}}
\sisetup{group-separator = {,}, group-digits = true}

%% -------------------- tikz --------------------
\usepackage{tikz}
\usetikzlibrary{positioning}
\usetikzlibrary{arrows,backgrounds,automata,decorations.shapes,decorations.pathmorphing,decorations.markings,decorations.text}

\tikzstyle{place}=[circle,draw=blue!50,fill=blue!20,thick, inner sep=0pt,minimum size=6mm]
\tikzstyle{transition}=[rectangle,draw=black!50,fill=black!20,thick, inner sep=0pt,minimum size=4mm]

\tikzstyle{block}=[rectangle,draw=black, thick, inner sep=5pt]
\tikzstyle{bullet}=[circle,draw=black, fill=black, thin, inner sep=2pt]

\tikzstyle{pre}=[<-,shorten <=1pt,>=stealth',semithick]
\tikzstyle{post}=[->,shorten >=1pt,>=stealth',semithick]
\tikzstyle{bi}=[<->,shorten >=1pt,shorten <=1pt, >=stealth',semithick]

\tikzstyle{mut}=[-,>=stealth',semithick]

\tikzstyle{treereset}=[dashed,->, shorten >=1pt,>=stealth',thin]

\usepackage{ifmtarg}
\usepackage{xifthen}
\makeatletter
% new counter to now which frame it is within the sequence
\newcounter{multiframecounter}
% initialize buffer for previously used frame title
\gdef\lastframetitle{\textit{undefined}}
% new environment for a multi-frame
\newenvironment{multiframe}[1][]{%
\ifthenelse{\isempty{#1}}{%
% if no frame title was set via optional parameter,
% only increase sequence counter by 1
\addtocounter{multiframecounter}{1}%
}{%
% new frame title has been provided, thus
% reset sequence counter to 1 and buffer frame title for later use
\setcounter{multiframecounter}{1}%
\gdef\lastframetitle{#1}%
}%
% start conventional frame environment and
% automatically set frame title followed by sequence counter
\begin{frame}%
\frametitle{\lastframetitle~{\normalfont(\arabic{multiframecounter})}}%
}{%
\end{frame}%
}
\makeatother

\makeatletter
\newdimen\tu@tmpa%
\newdimen\ydiffl%
\newdimen\xdiffl%
\newcommand\ydiff[2]{%
    \coordinate (tmpnamea) at (#1);%
    \coordinate (tmpnameb) at (#2);%
    \pgfextracty{\tu@tmpa}{\pgfpointanchor{tmpnamea}{center}}%
    \pgfextracty{\ydiffl}{\pgfpointanchor{tmpnameb}{center}}%
    \advance\ydiffl by -\tu@tmpa%
}
\newcommand\xdiff[2]{%
    \coordinate (tmpnamea) at (#1);%
    \coordinate (tmpnameb) at (#2);%
    \pgfextractx{\tu@tmpa}{\pgfpointanchor{tmpnamea}{center}}%
    \pgfextractx{\xdiffl}{\pgfpointanchor{tmpnameb}{center}}%
    \advance\xdiffl by -\tu@tmpa%
}
\makeatother
\newcommand{\copyrightbox}[3][r]{%
\begin{tikzpicture}%
\node[inner sep=0pt,minimum size=2em](ciimage){#2};
\usefont{OT1}{phv}{n}{n}\fontsize{4}{4}\selectfont
\ydiff{ciimage.south}{ciimage.north}
\xdiff{ciimage.west}{ciimage.east}
\ifthenelse{\equal{#1}{r}}{%
\node[inner sep=0pt,right=1ex of ciimage.south east,anchor=north west,rotate=90]%
{\raggedleft\color{black!50}\parbox{\the\ydiffl}{\raggedright{}#3}};%
}{%
\ifthenelse{\equal{#1}{l}}{%
\node[inner sep=0pt,right=1ex of ciimage.south west,anchor=south west,rotate=90]%
{\raggedleft\color{black!50}\parbox{\the\ydiffl}{\raggedright{}#3}};%
}{%
\node[inner sep=0pt,below=1ex of ciimage.south west,anchor=north west]%
{\raggedleft\color{black!50}\parbox{\the\xdiffl}{\raggedright{}#3}};%
}
}
\end{tikzpicture}
}


%% --------------------

%\usepackage[excludeor]{everyhook}
%\PushPreHook{par}{\setbox0=\lastbox\llap{MUH}}\box0}

%\vspace*{\stretch{1}

%\setbox0=\lastbox \llap{\textbullet\enskip}\box0}

\setlength{\parskip}{\fill}

\newcommand\noskips{\setlength{\parskip}{1ex}}
\newcommand\doskips{\setlength{\parskip}{\fill}}

\newcommand\xx{\par\vspace*{\stretch{1}}\par}
\newcommand\xxs{\par\vspace*{2ex}\par}
\newcommand\tuple[1]{\langle #1 \rangle}
\newcommand\code[1]{{\sf \footnotesize #1}}
\newcommand\ex[1]{\uline{Example:} \ifdefined \presentationonly \pause \fi
  \ifdefined\showexamples#1\xspace\else{\uline{\hspace*{2cm}}}\fi}

\newcommand\ceil[1]{\lceil #1 \rceil}


\AtBeginSection[]
{
   \begin{frame}
       \frametitle{Outline}
       \tableofcontents[currentsection]
   \end{frame}
}



\pgfdeclarelayer{edgelayer}
\pgfdeclarelayer{nodelayer}
\pgfsetlayers{edgelayer,nodelayer,main}

\tikzstyle{none}=[inner sep=0pt]
\tikzstyle{rn}=[circle,fill=Red,draw=Black,line width=0.8 pt]
\tikzstyle{gn}=[circle,fill=Lime,draw=Black,line width=0.8 pt]
\tikzstyle{yn}=[circle,fill=Yellow,draw=Black,line width=0.8 pt]
\tikzstyle{empty}=[circle,fill=White,draw=Black]
\tikzstyle{bw} = [rectangle, draw, fill=blue!20, 
    text width=4em, text centered, rounded corners, minimum height=2em]
    
    \newcommand{\CcNote}[1]{% longname
	This work is licensed under the \textit{Creative Commons #1 3.0 License}.%
}
\newcommand{\CcImageBy}[1]{%
	\includegraphics[scale=#1]{creative_commons/cc_by_30.pdf}%
}
\newcommand{\CcImageSa}[1]{%
	\includegraphics[scale=#1]{creative_commons/cc_sa_30.pdf}%
}
\newcommand{\CcImageNc}[1]{%
	\includegraphics[scale=#1]{creative_commons/cc_nc_30.pdf}%
}
\newcommand{\CcGroupBySa}[2]{% zoom, gap
	\CcImageBy{#1}\hspace*{#2}\CcImageNc{#1}\hspace*{#2}\CcImageSa{#1}%
}
\newcommand{\CcLongnameByNcSa}{Attribution-NonCommercial-ShareAlike}

\newenvironment{changemargin}[1]{% 
  \begin{list}{}{% 
    \setlength{\topsep}{0pt}% 
    \setlength{\leftmargin}{#1}% 
    \setlength{\rightmargin}{1em}
    \setlength{\listparindent}{\parindent}% 
    \setlength{\itemindent}{\parindent}% 
    \setlength{\parsep}{\parskip}% 
  }% 
  \item[]}{\end{list}} 




\title{Lecture 34 --- DevOps: Configuration }

\author{Patrick Lam \& Jeff Zarnett \\ \small \texttt{patrick.lam@uwaterloo.ca} \texttt{jzarnett@uwaterloo.ca}}
\institute{Department of Electrical and Computer Engineering \\
  University of Waterloo}
\date{\today}


\begin{document}

\begin{frame}
  \titlepage

 \end{frame}



\begin{frame}
\frametitle{DevOps for P4P}

\large

So far, one-off computations:
you need to answer a question, so you write code to do that.

But many systems are long-running (``generally available'').\\
$\Rightarrow$ Operations.

\end{frame}



\begin{frame}
\frametitle{Keep it Rolling}

\Large
Cloud computing: often long-lived systems,\\
but we didn't talk about how.\\[1em]
Today: many companies fuse\\
development (writes the software) \\
and operations (tends the software).

\end{frame}



\begin{frame}
\frametitle{Disaster Girl Strikes Again}

\begin{center}
	\includegraphics[width=0.8\textwidth]{images/devops.jpg}
\end{center}

\end{frame}



\begin{frame}
\frametitle{Start Me Up}

\Large

Startups:\\[1em]
No money to pay for separate \\
developer and operations teams.\\[1em]
Not that many servers, \\
just a few demo systems, test systems, etc\ldots\\
but it spirals out from there. \\[1em]
You're not really going to ask Sales to manage these servers, are you? \\
So, there's DevOps. 


\end{frame}



\begin{frame}
\frametitle{DevOps---Good Plan?}

\large

Is DevOps a good idea? \\
Can be used for both good and evil. \\[1em]
Good:
\begin{itemize}
\item developers involved across the software lifecycle.\\
(can learn a lot doing ops\ldots )
\item developers motivated to use correct tools \& document processes.
\end{itemize}


\end{frame}



\begin{frame}
\frametitle{Continuous Integration}

Each change or related group of changes is evaluated:

\begin{itemize}
	\item Pull from version Control
	\item Build
	\item Test
	\item Report results
\end{itemize}

Social convention to not break the build! Slack/Teams/etc. notifications.

\end{frame}


\begin{frame}
\frametitle{Configuration as Code}

\large

Systems have long come with \\
complicated (``flexible'') configuration options.


Sendmail is particularly notorious, but apache and nginx aren't super
easy to configure either.

First principle: treat \emph{configuration as code}.


\end{frame}



\begin{frame}
\frametitle{Configuration as Code}

\large

\begin{itemize}
\item use version control on your configuration.
\item implement code reviews on changes to the configuration.
\item test your configurations.
\item aim for a suite of modular services that integrate together smoothly.
\item refactor configuration files.
\item use continuous builds.
\end{itemize}


\end{frame}



\begin{frame}
\frametitle{Autoconfig}

\large

Excellent idea: tools for configuration. \\[1em]

Not enough to write text \\
\qquad ``How to Install AwesomeApp'' \\[1em]

e.g. use Terraform---\\
build, installation, and update automatic \& simple.\\[1em]

 Complicated means mistakes\ldots people forget steps. They are human. 
 
 
\end{frame}


\begin{frame}
\frametitle{Terraform}

\begin{center}
	\includegraphics[width=0.5\textwidth]{images/terraform.png}
\end{center}

Its whole purpose is to manage your config as codes
situation where you want to run your code using a cloud provider (e.g., AWS).

\end{frame}


\begin{frame}
\frametitle{Planning is Essential}

Terraform has a \emph{plan} operation: can verify the change it's about to make. 

Verify that we aren't about to give all our money to Jeff Bezos but also that a small change is actually small.

If you are happy with the change, \emph{apply} it---but things can change between plan and apply!

\end{frame}


\begin{frame}
\frametitle{Destructive Changes}

You can accidentally tell it you want to destroy all Github groups and people DM you thinking that this means they got fired.

\begin{center}
	\includegraphics[width=0.4\textwidth]{images/fired.jpg}
\end{center}

\end{frame}

\begin{frame}
\frametitle{Common Infrastructure}

\large

Use APIs to access your infrastructure. Examples:

\begin{itemize}
\item storage
\item naming and discovery
\item monitoring
\end{itemize}

Avoid one-offs---use open-source tools when applicable.\\
But build your own tools if needed.


\end{frame}

\begin{frame}
\frametitle{Oh, Think Twice...}

Is this what we are best at?

Think extra carefully if you plan to do roll-your-own anything that is security or encryption related.

Also, remember that platforms like AWS are constantly launching new tools.

\end{frame}

\begin{frame}
\frametitle{Naming}

\Large

Naming is one of the hard problems in computing. 

There are only two hard things in computers:
\begin{enumerate}
\item cache invalidation,
\item naming things, and
\item off by one errors.
\end{enumerate}

\end{frame}



\begin{frame}
\frametitle{Naming Suggestions}

\large

\begin{itemize}
\item use canonical one-word names for servers;
\item but, use aliases to specify functions: geography/environment/purpose/serial
\end{itemize}


There's also the Java package approach of infinite dots: live.application.customer.webdomain.com

Pick something and be consistent.

\end{frame}


\begin{frame}
\frametitle{Billing or Potato?}

Debates rage about whether names should be meaningful or fun.

If the service is called \texttt{billing} it may be helpful in determining what it does, more so than if it were called \texttt{potato}.

But when you say the word do you mean the service or the team?

What if we need a new billing service? \texttt{billing2}

\end{frame}

\begin{frame}[fragile]
\frametitle{New College, founded 1379}

\begin{center}
\includegraphics[width=.8\textwidth]{images/New_College_garden_front_Oxford_England.jpg}\\
CC-BY-SA 2.0, SnapshotsofthePast.com\\
\tiny \url{https://commons.wikimedia.org/wiki/File:New_College_garden_front_Oxford_England.jpg}
\end{center}
\end{frame}

\begin{frame}
\frametitle{Descriptive Names aren't Magic}

I've seen examples where the teams are called (anonymized a bit) ``X~infrastructure'' and ``X operations''.

I'd estimate that 35\% of queries to each team result in a reply that says that the question should go to the other team. 

It gets worse when a team is responsible for a common or shared component (e.g., a library). 

\end{frame}


\begin{frame}
\frametitle{Naming Solutions}

Real solution is like service discovery: tool with directory info.

Something like Opslevel exists; use it!

And don't forget that fun has a morale impact. 

\end{frame}

\begin{frame}
\frametitle{Servers as Cattle, not Pets}

\large

Servers means servers, or virtual machines, or containers.\\[1em]

At scale (smaller than you think):\\
use mass tools for dealing with servers, \\
rather than doing tasks manually. \\[1em]

At least: cloud-like server initialization without manual intervention;\\
must be able to spin up a server programmatically.


\end{frame}



\begin{frame}
\frametitle{Another Example}

\begin{center}
	\includegraphics[width=0.5\textwidth]{images/Kubernetes.jpg}
\end{center}

This is used to automate deploying and scaling of applications.

\end{frame}

\begin{frame}
\frametitle{Canarying}

\begin{center}
	\includegraphics[width=0.9\textwidth]{images/blackcanary.jpg}
\end{center}

\end{frame}



\begin{frame}
\frametitle{Canarying}

\large

Deploy new code incrementally in production, \\
also known as ``test in prod'':


\begin{itemize}
\item stage for deployment;
\item remove canary servers from service;
\item upgrade canary servers;
\item run automatic tests on upgraded canaries;
\item reintroduce canary servers into service;
\item see how it goes!
\end{itemize}

Of course: implement your system with rollback.


\end{frame}







\begin{frame}
\frametitle{Containerize Me, Captain}

\begin{center}
	\includegraphics[width=0.7\textwidth]{images/container.jpeg}
\end{center}

\end{frame}


\begin{frame}
\frametitle{How Did We Get Here?}

\begin{itemize}
	\item Manual install
	\item Package Manager (RPM/JAR/DLL Hell)
	\item Virtual Machines
	\item Containerization
\end{itemize}


\end{frame}


\begin{frame}
\frametitle{Containers}

See this diagram from NetApp:

\begin{center}
	\includegraphics[width=0.7\textwidth]{images/cvm.png}
\end{center}


\end{frame}





\end{document}

