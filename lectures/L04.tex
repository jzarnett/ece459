\documentclass[letterpaper,10pt]{article}

\usepackage{titling}
\usepackage{listings}
\usepackage{url}
\usepackage{setspace}
\usepackage{subfig}
\usepackage{sectsty}
\usepackage{pdfpages}
\usepackage{colortbl}
\usepackage{multirow}
\usepackage{multicol}
\usepackage{relsize}
\usepackage{amsmath}
\usepackage{fancyvrb}
\usepackage{amsmath,amssymb,amsthm,graphicx,xspace}
\usepackage[titlenotnumbered,noend,noline]{algorithm2e}
\usepackage[compact]{titlesec}
\usepackage{XCharter}
\usepackage[T1]{fontenc}
\usepackage{enumitem}
\usepackage{tikz}
\usetikzlibrary{arrows,automata,shapes,trees,matrix,chains,scopes,positioning,calc}
\tikzstyle{block} = [rectangle, draw, fill=blue!20, 
    text width=2.5em, text centered, rounded corners, minimum height=2em]
\tikzstyle{bw} = [rectangle, draw, fill=blue!20, 
    text width=4em, text centered, rounded corners, minimum height=2em]

\newcommand{\CPP}{C\nolinebreak\hspace{-.05em}\raisebox{.4ex}{\tiny\bf +}\nolinebreak\hspace{-.10em}\raisebox{.4ex}{\tiny\bf +}}
\def\CPP{{C\nolinebreak[4]\hspace{-.05em}\raisebox{.4ex}{\tiny\bf ++}}}

\let\LaTeXtitle\title
\renewcommand{\title}[1]{\LaTeXtitle{\textsf{#1}}}


\addtolength{\oddsidemargin}{-1.000in}
\addtolength{\evensidemargin}{-0.500in}
\addtolength{\textwidth}{2.0in}
\addtolength{\topmargin}{-1.000in}
\addtolength{\textheight}{1.75in}
\addtolength{\parskip}{\baselineskip}
\setlength{\parindent}{0in}
\renewcommand{\baselinestretch}{1.5}

\singlespace


\begin{document}

\lecture{4 --- Concurrency and Parallelism}{\term}{Jeff Zarnett, based on original by Patrick Lam}

\section*{Concurrency and Parallelism}
Concurrency and parallelism both give up the
total ordering between instructions in a sequential program, for
different purposes.

\paragraph{Concurrency.} We'll refer to the use of threads for
structuring programs as concurrency. Here, we're not aiming
for increased performance. Instead, we're trying to write the program
in a natural way. Concurrency makes sense as a model for distributed
systems, or systems where multiple components interact, with no ordering
between these components, like graphical user interfaces.

\paragraph{Parallelism.} We're studying parallelism in this class, where
we try to do multiple things at the same time in an attempt to increase
throughput. Concurrent programs may be easier to parallelize.

\section*{Processor Design Issues}
Recall that we listened to Cliff Click describe characteristics of
modern processors in Lecture 2. In this lecture we'll continue our
quick review of computer architecture and how it relates to
programming for performance. Here's another reference about chip
multi-threading; we are going to study some
of the techniques in the ``Writing Scalable Low-Level Code'' section of~\cite{extremescaling}:

\paragraph{Scalable Algorithms.} 
Remember from ECE~250 that we often care about the worst case run-time performance of the algorithm. An algorithm that's $O(n^{3})$ scales so much worse than one that's $O(n)$ that it's not even funny. Trying to do an insertion sort on a small array is fine (actually... recommended); doing it on a huge array is madness. Choosing a good algorithm is very important if we want it to scale. 

\begin{center}
	\includegraphics[width=0.75\textwidth]{images/big-o-complexity}\\
	Big-O Complexity comparison from ~\cite{bigocheatsheet}
\end{center}

\paragraph{Locking.}
Think back to the operating systems and systems programming course and the discussion of locking and concurrency. We'll be coming back to this subject before too long. But for now, suffice it to say, that the more locks and locking we need, the less scalable the code is going to be. You may think of the lock as a resource and the more threads or processes that are looking to acquire that lock, the more ``resource contention'' we have, and the more waiting and coordination are going to be necessary.

\paragraph{Cache Line Sharing.}
Multiprocessor (multicore) processors have some hardware that tries to keep the data consistent between different pipelines and caches (as we saw in the video). More processors, more threads means more work is necessary to keep these things in order.

\paragraph{Pools of Worker Threads.}
If we have a pool of workers, the application just submits units of work, and then on the other side these units of work are allocated to workers. The number of workers will scale based on the available hardware. This is neat as a programming practice: as the application developer we don't care quite so much about the underlying hardware. Let the operating system decide how many workers there should be, to figure out the optimal way to process the units of work.

\paragraph{Memory Allocators.}
Assuming we're not working with an embedded system where all memory is statically allocated in advance, there will be dynamic memory allocation. The memory allocator is often centralized and may support only one thread allocating or deallocating at a time. This means it does not necessarily scale very well, though there are some 


\section*{Processes and Threads.} Let's review the
difference between a process and a thread. A \emph{process} is an
instance of a computer program that contains program code and its own
address space, stack, registers, and resources (file handles, etc).  A
\emph{thread} usually belongs to a process. The most important point
is that it shares an address space with its parent process, hence
variables and code as well as resources. Each thread has its own~\cite{osi}:
\begin{enumerate}
	\item Thread execution state (like process state: running, ready, blocked...).
	\item Saved thread context when not running.
	\item Execution stack.
	\item Local variables.
	\item Access to the memory and resources of the process (shared with all threads in that process).
\end{enumerate}

Or, to represent this visually:

\begin{center}
	\includegraphics[width=0.625\textwidth]{images/mthread2.png}\\
	A single threaded and a multithreaded process compared side-by-side~\cite{osc}.
\end{center}

All the threads of a process share the state and resources of the process. If one thread opens a file, other threads in that process can also access that file.

The way programs are written now, there are few if any that are not in some way multithreaded. One common way of dividing up the program into threads is to separate the user interface from a time-consuming action. 

Consider a file-transfer program. If the user interface and upload method share a thread, once a file upload has started, the user will not be able to use the UI anymore (and Windows will put the dreaded ``(Not Responding)'' at the end of its dialog title), even to click the button that cancels the upload. For some reason, users hate that. 

We have two options for how to alleviate this problem: when an upload is ready to start, we can call \texttt{fork} and create a new process to do the upload, or we can spawn  new thread. In either case, the newly created entity will handle the upload of the file. The UI remains responsive, because the UI thread is not waiting for the upload method to complete.

\subsection*{Motivation for Threads}

Why choose threads rather than creating a new process? The primary, but not sole, motivation is performance:
\begin{enumerate}
	\item Creating a new thread is much faster than creating a new process. In fact, thread creation is on the order of ten times faster~\cite{machThreads}.
	\item Terminating and cleaning up a thread is faster than terminating and cleaning up a process.
	\item It takes less time to switch between two threads within the same process (because less data needs to be stored/restored). In Solaris, for example, switching between processes is about five times slower than switching between threads~\cite{osc}.
	\item Because threads share the same memory space, for two threads to communicate, they do not have to use any of the IPC mechanisms; they can just communicate directly.
	\item As in the file transfer program, use of threads allows the program to be responsive even when a part of the program is blocked.
\end{enumerate}

This last advantage, background work, is one of four common examples of the uses of threads in a general purpose operating system~\cite{insideOS2}:
\begin{enumerate}
	\item \textbf{Foreground and Background Work:} as already examined, the ability to run something in the background to keep the program responsive.
	\item \textbf{Asynchronous processing}: for example, to protect against power failure or a crash, a word processor may write the document data in main memory to disk periodically. This can be done as a background task so it does not disrupt the user's workflow.
	\item \textbf{Speed of Execution:} a multithreaded program can get more done in the same amount of time. Just as the OS can run a different program when the executing program gets blocked (say, on a disk read), if one thread is blocked, another thread may execute.
	\item \textbf{Modular Structure:} a program that does several different things may be given structure through threads.
\end{enumerate}

There are some drawbacks, however: there is no protection between threads in the same process: so one thread can easily mess with the memory being used by another thread. This once again brings us to the subject of co-ordination, which will follow the discussion of threads.

Also, if any thread encounters an error (such as a division by zero or Segmentation Fault), the whole process might be terminated by the operating system. If the program has multiple processes for different parts, then the other processes will not be affected; if the program has multiple threads and they all share the same process, then any thread encountering an error might bring all of them to a halt.



You can find another explanation of processes versus threads here:
\begin{center}
  \url{http://www.purplealienplanet.com/node/50}
\end{center}

\paragraph{Threads and CPUs.} In your operating systems class,
you've seen implementations of threads (``lightweight processes'').
We'll call these threads \emph{software threads}, and we'll program
with them throughout the class. Each software thread corresponds to a
stream of instructions that the processor executes. On a old-school
single-core, single-processor machine, the operating system
multiplexes the CPU resources to execute multiple threads
concurrently; however, only one thread runs at a time on the single
CPU.

On the other hand, a modern chip contains a number of \emph{hardware
threads}, which correspond to the virtual CPUs. These are sometimes
known as \emph{strands}. The operating system still needs to 
multiplex the software threads onto the hardware threads, but now has
more than one hardware thread to schedule work onto.



\paragraph{Implementing (or Simulating) Hardware Threads.} 
There are a number of ways to implement multiple software threads; for
instance, the simplest possible implementation, {\bf kernel-level
threading} (or 1:1 model) dedicates one core to each thread. The kernel schedules threads
on different processors. (Note that kernel involvement will always be required to
take advantage of a multicore system). This model is used by Win32, as well as
POSIX threads for Windows and Linux. The 1:1 model allows concurrency and parallelism.

Alternately, we could make one core execute multiple threads, in the
{\bf user-level threading}, or N:1, model. The single core would keep
multiple contexts and could 1) switch every 100 cycles; 2) switch
every cycle; 3) fetch one instruction from each thread each cycle; or
4) switch every time the current thread hits a long-latency event
(cache miss, etc.) This model allows for quick context switches, but
does not leverage multiple processors. (Why would you use these?) The N:1 model is used by
GNU Portable Threads.

Finally, it's possible to both use multiple cores and put multiple
threads onto one core, in a {\bf hybrid threading}, or M:N, model. Here, we map $M$ application threads to $N$ kernel threads. This
is a compromise between the previous two models, which both allows
quick context switches and the use of multiple processors. However, it
requires increased complexity; the library provides scheduling
services, which may not coordinate well with kernel, and increases
likelihood of priority inversion (which you've seen in Operating
Systems).  This method is used by modern Windows threads.



%\paragraph{Live Coding Example.} We can write a program to determine
%whether we are in a user-level threading model vs. the other two models.
%Just start a child thread and make both threads compute. If CPU utilization
%exceeds 100\%, we are not user-level threading. The example I saw in class
%computes $2^{28}$ random numbers to spin the CPU.

%(Is it possible to differentiate hybrid and kernel threading by writing a
%simple program?)



\bibliographystyle{alpha}
\bibliography{459}


\end{document}
